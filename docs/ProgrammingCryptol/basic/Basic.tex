\chapter{Basic Examples}
\label{cha:basic-examples}

To be written.

\todo[inline]{Hello, world Cryptol equivalent.}

\todo[inline]{Counting.}

\todo[inline]{Fib.}

\todo[inline]{dumb sorting, bitonic sort}

\todo[inline]{Carefully discuss function application semantics and how
  they differ from what mathematicians are used to.  Relevant to
  \href{https://www.galois.com/cryptol/ticket/280}{ticket \#280}.}

\todo[inline]{Existing blogs: Legato, NQueens, Sudoku, substitution
  ciphers, riffle shuffle}

\todo[inline]{Existing crypto in blogs/text: MD6, BASE64, SIMON and SPECK,
  ZUC, Skein, the rail fence cipher}

\todo[inline]{Crypto we need to do: MD1, MD2, MD3, MD4, MD5}

\todo[inline]{Other fun stuff: (that hasn't been written about):
  U2Bridge crossing, foxGooseBeans, prolog}
        
\todo[inline]{Hamming encryption?}

\todo[inline]{fixed table lookup compression, run-length encoding,
  Huffman encoding}

\todo[inline]{searching on finite datatypes and infinite streams}

\todo[inline]{integer FFT}

\todo[inline]{put in something about cordic/floating
  point\indFloatingPoint}

\todo[inline]{Fix all uses of Example references using \texttt{autoref}.}

%%% Local Variables: 
%%% mode: latex
%%% TeX-master: "../main/Cryptol"
%%% End: 
