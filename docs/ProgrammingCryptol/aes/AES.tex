\commentout{
\begin{code}
  module AES where

  import Enigma
  // import Classic
\end{code}
}

%#####################################################################
\chapter{AES: The Advanced Encryption Standard}
\label{chapter:aes}

AES\indAES is a symmetric key encryption algorithm (a symmetric
cipher, per the discussion in~\autoref{chapter:classic}), based on the
Rijndael algorithm designed by Joan Daemen and Vincent
Rijmen~\cite{DaemenR02}\glosAES. (The term {\em symmetric key} means
that the algorithm uses the same key for encryption and
decryption.\indSymKey) AES was adopted in 2001 by the US government,
deemed suitable for protecting classified information up to {\em
  secret} level for the key size 128, and up to the {\em top-secret}
level for key sizes 192 and 256.

In this chapter, we will program AES in Cryptol. Our emphasis will be
on clarity, as opposed to efficiency, and we shall follow the NIST
standard description of AES\indAES fairly closely~\cite{aes}\glosNIST.
Referring to the standard as you work your way through this chapter is
recommended.

Some surprises may be at hand for the reader who has never deeply
examined a modern cryptography algorithm before.

First, algorithms like AES are typically composed of many smaller
units of varying kinds.  Consequently, the entire algorithm is
constructed bottom-up by specifying and verifying each of its
component pieces.  It is wise to handle smaller and simpler components
first.  It is also a good practice, though hard to accomplish the
first one or two times you write such a specification, to write
specifications with an eye toward reuse in multiple instantiations of
the same algorithm (e.g., different key sizes or configurations).  The
choice between encoding configurations at the type level or the value
level is aesthetic and practical: some verification is only possible
when one encodes information at the type level.

Second, algorithms frequently depend upon interesting data structures
and mathematical constructs, the latter of which can be though of as
data structures in a pure mathematics sense.  The definition, shape,
form, and subtleties of these data structures are critical to the
\emph{correct definition} of the crypto algorithm \emph{as well as its
  security properties}.  Implementing an algorithm using an
alternative datatype construction that you believe has the same
properties as that which is stipulated in a standard is nearly always
the wrong thing to do.  Also, the subtleties of these constructions
usually boils down to what an engineer might think of as ``magic
numbers''---strange initial values or specific polynomials that appear
out of thin air.  Just remind yourself that the discovery and analysis
of those magic values was, in general, the joint hard work of a
community of cryptographers.

%=====================================================================
\section{Parameters}
\label{sec:aesparams}

The AES algorithm always takes 128 bits of input, and always produces
128 bits of output, regardless of the key size.  The key-size can be
one of 128 (AES128), 192 (AES192), or 256 (AES256).  Following the
standard, we define the following three parameters~\cite[Section
2.2]{aes}:
\begin{itemize}
\item {\tt Nb}: Number of columns, always set to 4 by the standard.
\item {\tt Nk}: Number of key-blocks, which is the number of 32-bit
  words in the key: 4 for AES128, 6 for AES192, and 8 for AES256;
\item {\tt Nr}: Number of rounds, which {\tt Nr} is always $6 +
  \mbox{\tt Nk}$, according to the standard.  Thus, 10 for AES128, 12
  for AES192, and 14 for AES256.
\end{itemize}
The Cryptol definitions follow the above descriptions verbatim:
\begin{code}
  type AES128 = 4
  type AES192 = 6
  type AES256 = 8

  type Nk = AES128
  type Nb = 4
  type Nr = 6 + Nk
\end{code}
The following derived type is helpful in signatures:
\begin{code}
  type AESKeySize = (Nk*32)
\end{code}

%=====================================================================
% \section{Polynomials in \texorpdfstring{GF($2^8$)}{GF(2,8)}}
% \label{sec:polynomials}
\sectionWithAnswers{Polynomials in \texorpdfstring{GF($2^8$)}{GF(2,8)}}{sec:polynomials}

AES\indAES works on a two-dimensional representation of the input
arranged into bytes, called the {\em state}.\indAESState For a
128-bit input, we have precisely 4 rows, each containing {\tt Nb}
(i.e., 4) bytes, each of which is 8 bits wide, totaling
$4\times4\times8 = 128$ bits. The bytes themselves are treated as
finite field elements in the Galois field
GF($2^8$)~\cite{wiki:galoisfield}\indGF, giving rise to the following
declarations:
\begin{code}
  type GF28  = [8]
  type State = [4][Nb]GF28
\end{code}

The hard-encoding of \texttt{GF28} in this specification is completely
appropriate because the construction of AES depends entirely upon the
Galois field GF($2^8$).  It is conceivable that other algorithms might
be parameterized across GF($2^k$) for some $k$, in which case the
underlying type declaration would also be parameterized.

While a basic understanding Galois field operations is helpful, the
details are not essential for our modeling purposes.  It suffices to
note that GF($2^8$) has precisely 256 elements, each of which is a
polynomial\indPoly of maximum degree 7, where the coefficients are
either 0 or 1.  The numbers from 0 to 255 (i.e., all possible 8-bit
values) are in one-to-one correspondence with these polynomials.  The
coefficients of the polynomial come from the successive bits of the
number, and vice versa.  For instance the 8-bit number 87 can be
written as {\tt 0b01010111} in binary, and hence corresponds to the
polynomial $x^6 + x^4 + x^2 + x^1 + 1$. Similarly, the polynomial $x^4
+ x^3 + x$ corresponds to the number {\tt 0b00011010}, i.e., 26. We
can also compute this value by evaluating the polynomial for $x=2$,
obtaining $2^4+2^3+2 = 16+8+2 = 26$.

Cryptol allows you to write polynomials in GF($2^n$), for arbitrary
$n$, using the following notation:
\begin{Verbatim}
  Cryptol> <| x^^6 + x^^4 + x^^2 + x^^1 + 1 |>
  87
  Cryptol> 0b1010111
  87
  Cryptol> <| x^^4 + x^^3 + x |>
  26
  Cryptol> 0b11010
  26
\end{Verbatim}

A polynomial is similar to a decimal representation of a number,
albeit in a more suggestive syntax.  Like with a decimal, the Cryptol
type system will default the type to be the smallest number of bits
required to represent the polynomial, but it may be expanded to more
bits if an expression requires it.

\paragraph*{Addition and Subtraction} Given two polynomials,
adding\indAddPoly and subtracting\indSubPoly them in a Galois field
GF($2^n$) results in a new polynomial where terms with the same power
cancel each other out. When interpreted as a word, both addition and
subtraction amount to a simple exclusive-or operation.  Cryptol's
\Verb|^|\indXOr operator captures this idiom concisely:
\begin{Verbatim}
  Cryptol> <| x^^4 + x^^2 |> ^ <| x^^5 + x^^2 + 1 |> == \
           <| x^^5 + x^^4 + 1 |>
  True
\end{Verbatim}
Note that the term $x^2$ cancels since it appears in both polynomials.
\begin{Exercise}\label{ex:gf:0}
  Adding a polynomial to itself in GF($2^n$) will always yield {\tt 0}
  since all the terms will cancel each other.  Write and prove a
  theorem {\tt polySelfAdd} over {\tt GF28} to state this fact.
\end{Exercise}
\begin{Answer}\ansref{ex:gf:0}
\todo[inline]{Recheck use of parentheses in property after \ticket{208}
    is closed.}
\begin{code}
  polySelfAdd: GF28 -> Bit
  property polySelfAdd x = (x ^ x) == zero
\end{code}
We have:\indSBV
\begin{Verbatim}
  Cryptol> :prove polySelfAdd
  Q.E.D.
\end{Verbatim}
\end{Answer}
\unparagraph While adding two polynomials does not warrant a separate
function, we will need a version that can add a sequence of
polynomials:
\begin{Exercise}\label{ex:gf:add0}
Define a function
\begin{code}
  gf28Add : {n} (fin n) => [n]GF28 -> GF28
\end{code}
that adds all the elements given.
\lhint{Use a fold, see Pg.~\pageref{par:fold}.}\indFold
\end{Exercise}
\begin{Answer}\ansref{ex:gf:add0}
\begin{code}
  gf28Add ps = sums ! 0
    where sums = [zero] # [ p ^ s | p <- ps
                                  | s <- sums
                          ]
\end{code}
\end{Answer}

\paragraph*{Multiplication} Multiplication in GF($2^n$) follows the usual
polynomial multiplication algorithm, where we multiply the first
polynomial with each term of the second, and add all the partial sums
(i.e., compute their exclusive-or). While this operation can be
programmed explicitly, Cryptol does provide the primitive {\tt
  pmult}\indMulPoly for this purpose:
\begin{Verbatim}
  Cryptol> pmult <| x^^3 + x^^2 + x + 1 |> <| x^^2 + x + 1 |>
  45
  Cryptol> <| x^^5 + x^^3 + x^^2 + 1 |>
  45
\end{Verbatim}

\begin{Exercise}\label{ex:gf:1}
  Multiply the polynomials $x^3 +x^2+x+1$ and $x^2+x+1$ by hand in
  GF($2^8$) and show that the result is indeed $x^5+x^3+x^2+1$, (i.e.,
  $45$), justifying Cryptol's result above.
\end{Exercise}
\begin{Answer}\ansref{ex:gf:1}
  \todo[inline]{Figure out outdent on this answer if this todo note is
    removed.}  We first compute the results of multiplying our first
  polynomial ($x^3 +x^2+x+1$) with each term in the second polynomial
  ($x^2+x+1$) separately:
\begin{eqnarray*}
(x^3 + x^2 + x + 1) \times x^2   &=& x^5 + x^4 + x^3 + x^2 \\
(x^3 + x^2 + x + 1) \times x\;\; &=& x^4 + x^3 + x^2 + x \\
(x^3 + x^2 + x + 1) \times 1\;\; &=& x^3 + x^2 + x\;\; + 1
\end{eqnarray*}
We now add the resulting polynomials, remembering the adding the same
powered terms cancel each other out. For instance, we have two
instances each of $x^4$, $x^2$, and $x$, which all get canceled. We
have three instances each of $x^3$ and $x^2$, so they survive the
addition, etc.  After the addition, we are left with the polynomial
$x^5 + x^3 + x^2 + 1$, which can be interpreted as {\tt 0b00101101},
i.e., $45$.
\end{Answer}

\paragraph*{Reduction} If you multiply two polynomials with degrees $m$
and $n$, you will get a new polynomial of degree $m+n$. As we
mentioned before, the polynomials in GF($2^8$)\indGF have degree at
most 7. Obviously, $m+n$ can be larger than $7$ when we multiply to
elements of GF($2^8$).  So we have to find a way to map the resulting
larger-degree polynomial back to an element of GF($2^8$). This is done
by reduction, or modulus, with respect to an {\em irreducible
  polynomial}\indIrredPoly. The AES algorithm uses the following
polynomial for this purpose:
\begin{code}
  irreducible = <| x^^8 + x^^4 + x^^3 + x + 1 |>
\end{code}
(Recall in the introduction of this chapter our warning about magic!)

Note that {\tt irreducible} is {\em not} an element of GF($2^8$),
since it has degree 8. However we can use this polynomial to define
the multiplication routine itself, which uses Cryptol's {\tt
  pmod}\indModPoly (polynomial modulus) function, as follows:
\begin{code}
  gf28Mult : (GF28, GF28) -> GF28
  gf28Mult (x, y) = pmod (pmult x y) irreducible
\end{code}
Polynomial modulus and division operations follow the usual schoolbook
algorithm for long-division---a fairly laborious process in itself,
but it is well studied in mathematics~\cite{wiki:polydiv}. Luckily for
us, Cryptol's {\tt pdiv}\indDivPoly and {\tt pmod}\indModPoly
functions implement these operations, saving us the programming task.

\begin{Exercise}\label{ex:pdivmod}
  Divide $x^5 + x^3 + 1$ by $x^3 + x^2 + 1$ by hand, finding the
  quotient and the remainder.  Check your answer with Cryptol's {\tt
    pmod} and {\tt pdiv} functions.
\end{Exercise}
\begin{Answer}\ansref{ex:pdivmod}
The long division algorithm is laborious, but not particularly hard:
\[
\renewcommand\arraystretch{1.2}
\begin{array}{*1r @{\hskip\arraycolsep}c@{\hskip\arraycolsep} *6r}
%  &    &&    &  1 &  1 & 0 \\
                &     &      &      &       & x^2   &+ x &    \\
\cline{2-8}
  x^3 + x^2 + 1 &\big)& x^5 &      & + x^3 &       &    & + 1\\
                  & & x^5 & + x^4 &      & + x^2  &    &           \\
\cline{3-6}
                  & &     & x^4  & + x^3 & + x^2 &    & + 1  \\
                  & &     & x^4  & + x^3 &       & + x&        \\
\cline{4-8}
                  & &     &      &       & x^2   & + x & + 1          \\
\end{array}
\]
Therefore, the quotient is $x^2+x$ and the remainder is $x^2+x+1$. We
can verify this easily with Cryptol:
\begin{Verbatim}
  Cryptol> pdiv <| x^^5 + x^^3 + 1 |> <| x^^3 + x^^2 + 1 |> == \
     <| x^^2 + x |>
  True
  Cryptol> pmod <| x^^5 + x^^3 + 1 |> <| x^^3 + x^^2 + 1 |> == \
     <| x^^2 + x + 1 |>
  True
\end{Verbatim}
Another way to check your result would be to multiply the quotient by
the divisor and add the remainder, and check that it gives us
precisely the polynomial we started with:
\begin{Verbatim}
  Cryptol> pmult <| x^^2 + x |>  <| x^^3 + x^^2 + 1 |> \
           ^ <| x^^2 + x + 1 |>
  0x29
  Cryptol> <| x^^5 + x^^3 + 1 |>
  0x29
\end{Verbatim}
\end{Answer}

\begin{Exercise}\label{ex:gf:2}
  Write and prove theorems showing that {\tt gf28Mult} (i) has the
  polynomial 1 as its unit, (ii) is commutative, and (iii) is
  associative.
\end{Exercise}
\begin{Answer}\ansref{ex:gf:2}
\fvset{fontsize=\footnotesize}
\begin{code}
  property gf28MultUnit         x =      gf28Mult(x, 1) == x
  property gf28MultCommutative  x y =    gf28Mult(x, y) == gf28Mult(y, x)
  property gf28MultAssociative  x y z =  gf28Mult(x, gf28Mult(y, z))
                                         == gf28Mult(gf28Mult(x, y), z)
\end{code}
It turns out that proving the unit and commutativity are fairly
trivial:\indCmdProve
\begin{Verbatim}
  Cryptol> :prove gf28MultUnit 
  Q.E.D.
  Cryptol> :prove gf28MultCommutative
  Q.E.D.
\end{Verbatim}
But {\tt aesMultAssociative} takes much longer! We show the results of
{\tt :check} below:\indCmdCheck
\begin{Verbatim}
  Cryptol> :check gf28MultAssociative
  Checking case 1000 of 1000 (100.00%) 
  1000 tests passed OK
\end{Verbatim}
%%   [Coverage: 5.96e-3%. (1000/16777216)]
Note that the coverage is pretty small (on the order of $0.006\%$)
in this case. Proving associativity of multiplication algorithms using
SAT/SMT based technologies is a notoriously hard task~\cite[Section
6.3.1]{DecisionProcedures2008}. If you have the time, you can let
Cryptol run long enough to complete the {\tt :prove
  gf28MultAssociative} command, however.
\end{Answer}

%=====================================================================
% \section{The {\ttfamily{\textbf SubBytes}} transformation}
% \label{aes:subbytes}
\sectionWithAnswers{The {\ttfamily{\textbf SubBytes}} transformation}{aes:subbytes}

\todo[inline]{Introduce a figure here, perhaps lifted from the
  standard, to better explain this stage of the algorithm.}

Recall that the state in AES\indAES is a $4\times4$ matrix of
bytes. As part of its operation, AES\indAES performs the so called
{\tt SubBytes} transformation\indAESSbox~\cite[Section 5.1.1]{aes},
substituting each byte in the state with another element. Given an $x
\in \mbox{GF}(2^8)$,\indGF the substitution for $x$ is computed as
follows:
\begin{enumerate}
\item Compute the multiplicative inverse of $x$ in GF$(2^8)$\indGF,
  call it $b$. If $x$ is 0, then take 0 as the result.
 \item Replace bits in $b$ as follows: Each bit $b_i$ becomes $b_i
   \oplus b_{i+4\imod{8}} \oplus b_{i+5\imod{8}} \oplus
   b_{i+6\imod{8}} \oplus b_{i+7\imod{8}} \oplus c_i$.  Here $\oplus$
   is exclusive-or and $c$ is {\tt 0x63}.
\end{enumerate}

\paragraph*{Computing the multiplicative inverse} It turns out that the
inverse of any non-zero $x$ in GF$(2^8)$ can be computed by raising
$x$ to the power 254, i.e., multiplying $x$ by itself 254
times. (Mathematically, GF$(2^8)$\indGF is a cyclic group such that
$x^{255}$ is always $1$ for any $x$, making $x^{254}$ the
multiplicative inverse of $x$.)

\begin{Exercise}\label{ex:gfmi:0}
Write a function
\begin{code}
  gf28Pow : (GF28, [8]) -> GF28
\end{code}
such that the call {\tt gf28Pow (n, k)} returns $n^k$ using {\tt
  gf28Mult} as the multiplication operator.  \lhint{Use the fact that
  $x^0 = 1$, $x^{2n} = (x^n)^2$, and $x^{2n+1} = x \times (x^n)^2$ to
  speed up the exponentiation.}
\end{Exercise}
\begin{Answer}\ansref{ex:gfmi:0}
\begin{code}
  gf28Pow (n, k) = pow k
     where   sq x  = gf28Mult (x, x)
             odd x = x ! 0
             pow i = if i == 0 then 1
                     else if odd i
                          then gf28Mult (n, sq (pow (i >> 1)))
                          else sq (pow (i >> 1))
\end{code}
Here is a version that follows the stream-recursion pattern:
\begin{code}
  gf28Pow' : (GF28, [8]) -> GF28
  gf28Pow' (n, k) = pows ! 0
    where   pows = [1] # [ if bit then gf28Mult (n, sq x)
                                  else sq x
                         | x <- pows
                         | bit <- k
                         ]
            sq x = gf28Mult (x, x)
\end{code}
\end{Answer}

\begin{Exercise}\label{ex:gfmi:01}
Write a function
\begin{code}
  gf28Inverse : GF28 -> GF28
\end{code}
to compute the multiplicative inverse of a given element by raising it
to the power $254$. Note that {\tt gf28Inverse} must map $0$ to
$0$. Do you have to do anything special to make sure this happens?
\end{Exercise}
\begin{Answer}\ansref{ex:gfmi:01}
\begin{code}
  gf28Inverse x = gf28Pow (x, 254)
\end{code}
We do not have to do anything special about $0$, since our {\tt
  gf28Inverse} function yields $0$ in that case:
\begin{Verbatim}
  Cryptol> gf28Inverse 0
  0x00
\end{Verbatim}
\end{Answer}

\begin{Exercise}\label{ex:gfmi:1}
  Write and prove a property {\tt gf28InverseCorrect}, ensuring that
  {\tt gf28Inverse x} does indeed return the multiplicative inverse of
  {\tt x}.  Do you have to do anything special when {\tt x} is $0$?
\end{Exercise}
\begin{Answer}\ansref{ex:gfmi:1}
  Since $0$ does not have a multiplicative inverse, we have to write a
  conditional property:
\begin{code}
  property gf28InverseCorrect x =
      if x == 0 then x' == 0 else gf28Mult(x, x') == 1
    where x' = gf28Inverse x
\end{code}
We have:
\begin{Verbatim}
  Cryptol> :prove gf28InverseCorrect
  Q.E.D.
\end{Verbatim}
\end{Answer}

\paragraph*{Transforming the result} As we mentioned above, the
AES\indAES specification asks us to transform each bit $b_i$ according
to the following transformation:
$$
 b_i \oplus b_{i+4\imod{8}} \oplus b_{i+5\imod{8}} \oplus b_{i+6\imod{8}} \oplus b_{i+7\imod{8}} \oplus c_i
$$
For instance, bit $b_5$ becomes $b_5 \oplus b_1 \oplus b_2 \oplus b_3
\oplus c_5$.  When interpreted at the word level, this basically
amounts to computing:
$$
 b \oplus (b \ggg 4) \oplus (b \ggg 5) \oplus (b \ggg 6) \oplus (b \ggg 7) \oplus c
$$
by aligning the corresponding bits in the word representation.

\todo[inline]{Justify/prove the equivalence of these two terms.
  Perhaps add a property to prove it in Cryptol?}

\begin{Exercise}\label{ex:aessbytes:0}
Write a function
\begin{code}
  xformByte : GF28 -> GF28
\end{code}
that computes the above described transformation.
\end{Exercise}
\begin{Answer}\ansref{ex:aessbytes:0}
\begin{code}
  xformByte b = gf28Add [b, (b >>> 4), (b >>> 5),
                         (b >>> 6), (b >>> 7), c]
    where c = 0x63
\end{code}
\end{Answer}

\todo[inline]{Remind the reader of the context again before putting
  everything together.}

\paragraph*{Putting it together} Armed with {\tt gf28Inverse} and {\tt
  xformByte}, we can easily code the function that transforms a single
byte as follows:
\begin{code}
  SubByte : GF28 -> GF28
  SubByte b = xformByte (gf28Inverse b)
\end{code}
AES's {\tt SubBytes}\indAES transformation merely applies this
function to each byte of the state:
\begin{code}
  SubBytes : State -> State
  SubBytes state = [ [ SubByte b | b <- row ] | row <- state ]
\end{code}

\paragraph*{Table lookup} Our definition of the {\tt SubByte} function
above follows how the designers of AES came up with the substitution
maps, i.e., it is a {\em reference} implementation. For efficiency
purposes, however, we might prefer a version that simply performs a
look-up in a table. Notice that the type of {\tt SubByte} is {\tt GF28
  -> GF28}, i.e., it takes a value between 0 and 255. Therefore, we
can make a table containing the precomputed values for all possible
256 inputs, and simply perform a table look-up instead of computing
these values each time we need it.  In fact, Figure~7 on page 16 of
the AES\indAES standard lists these precomputed values for
us~\cite[Section 5.1.1]{aes}. We capture this table below in Cryptol:
\todo[inline]{We should consistently use either ``look-up'' or ``lookup''.}

\vspace{0.25cm}
\begin{minipage}{\textwidth}
{\footnotesize
\begin{code}
  sbox : [256]GF28
  sbox = [
     0x63, 0x7c, 0x77, 0x7b, 0xf2, 0x6b, 0x6f, 0xc5, 0x30, 0x01, 0x67,
     0x2b, 0xfe, 0xd7, 0xab, 0x76, 0xca, 0x82, 0xc9, 0x7d, 0xfa, 0x59,
     0x47, 0xf0, 0xad, 0xd4, 0xa2, 0xaf, 0x9c, 0xa4, 0x72, 0xc0, 0xb7,
     0xfd, 0x93, 0x26, 0x36, 0x3f, 0xf7, 0xcc, 0x34, 0xa5, 0xe5, 0xf1,
     0x71, 0xd8, 0x31, 0x15, 0x04, 0xc7, 0x23, 0xc3, 0x18, 0x96, 0x05,
     0x9a, 0x07, 0x12, 0x80, 0xe2, 0xeb, 0x27, 0xb2, 0x75, 0x09, 0x83,
     0x2c, 0x1a, 0x1b, 0x6e, 0x5a, 0xa0, 0x52, 0x3b, 0xd6, 0xb3, 0x29,
     0xe3, 0x2f, 0x84, 0x53, 0xd1, 0x00, 0xed, 0x20, 0xfc, 0xb1, 0x5b,
     0x6a, 0xcb, 0xbe, 0x39, 0x4a, 0x4c, 0x58, 0xcf, 0xd0, 0xef, 0xaa,
     0xfb, 0x43, 0x4d, 0x33, 0x85, 0x45, 0xf9, 0x02, 0x7f, 0x50, 0x3c,
     0x9f, 0xa8, 0x51, 0xa3, 0x40, 0x8f, 0x92, 0x9d, 0x38, 0xf5, 0xbc,
     0xb6, 0xda, 0x21, 0x10, 0xff, 0xf3, 0xd2, 0xcd, 0x0c, 0x13, 0xec,
     0x5f, 0x97, 0x44, 0x17, 0xc4, 0xa7, 0x7e, 0x3d, 0x64, 0x5d, 0x19,
     0x73, 0x60, 0x81, 0x4f, 0xdc, 0x22, 0x2a, 0x90, 0x88, 0x46, 0xee,
     0xb8, 0x14, 0xde, 0x5e, 0x0b, 0xdb, 0xe0, 0x32, 0x3a, 0x0a, 0x49,
     0x06, 0x24, 0x5c, 0xc2, 0xd3, 0xac, 0x62, 0x91, 0x95, 0xe4, 0x79,
     0xe7, 0xc8, 0x37, 0x6d, 0x8d, 0xd5, 0x4e, 0xa9, 0x6c, 0x56, 0xf4,
     0xea, 0x65, 0x7a, 0xae, 0x08, 0xba, 0x78, 0x25, 0x2e, 0x1c, 0xa6,
     0xb4, 0xc6, 0xe8, 0xdd, 0x74, 0x1f, 0x4b, 0xbd, 0x8b, 0x8a, 0x70,
     0x3e, 0xb5, 0x66, 0x48, 0x03, 0xf6, 0x0e, 0x61, 0x35, 0x57, 0xb9,
     0x86, 0xc1, 0x1d, 0x9e, 0xe1, 0xf8, 0x98, 0x11, 0x69, 0xd9, 0x8e,
     0x94, 0x9b, 0x1e, 0x87, 0xe9, 0xce, 0x55, 0x28, 0xdf, 0x8c, 0xa1,
     0x89, 0x0d, 0xbf, 0xe6, 0x42, 0x68, 0x41, 0x99, 0x2d, 0x0f, 0xb0,
     0x54, 0xbb, 0x16]
\end{code}
}
\end{minipage}
\vspace{0.25cm}

\noindent With this definition of {\tt sbox}, the look-up variants of
{\tt SubByte} and {\tt SubBytes} becomes:\label{aes:subbytetl}
\begin{code}
  SubByte' : GF28 -> GF28
  SubByte' x = sbox @ x

  SubBytes' : State -> State
  SubBytes' state = [ [ SubByte' b | b <- row ] | row <- state ]
\end{code}

\begin{Exercise}\label{ex:sbox}
  Write and prove a property stating that {\tt SubByte} and 
    {\tt SubByte'} are equivalent.
\end{Exercise}
\begin{Answer}\ansref{ex:sbox}
\begin{code}
  property SubByteCorrect x = SubByte x == SubByte' x
\end{code}
We have:
\begin{Verbatim}
  Cryptol> :prove SubByteCorrect
  Q.E.D.
\end{Verbatim}
\end{Answer}

\note{The {\tt SubByte'} and {\tt SubBytes'} versions are going to be
  more efficient for execution, naturally. We should emphasize that
  this mode of development is quite common in modern cryptography.
  Ciphers are typically designed using ideas from mathematics, often
  requiring complicated algorithms.  To speed things up, however,
  implementors use clever optimization tricks, convert functions to
  look-up tables using precomputed values, etc.

  What Cryptol allows us to do is to write the algorithms using both
  styles, and then formally show that they are indeed equivalent, as
  you did in Exercise~\ref{ex:sbox} above.  This mode of
  high-assurance development makes sure that we have not made any
  cut-and-paste errors when we wrote down the numbers in our {\tt
    sbox} table. Equivalently, our proof also establishes that the
  official specification~\cite[Section 5.1.1]{aes} got its own table
  correct!}

\todo[inline]{Revisit a commented-out exercise here after \ticket{210}
  on \texttt{:safe} is addressed.}
% \begin{Exercise}\label{ex:sbox:safe}
%   The implementation of {\tt SubByte'} uses indexing ({\tt
%   @})\indIndex, and hence we might be worried about
%   index-out-of-bounds errors. In this case the table has precisely
%   256 elements and we are indexing with a byte, so the indexing must
%   be safe.  Use Cryptol's {\tt :safe} command\indCmdSafe to verify
%   this claim.
% \end{Exercise}
% \begin{Answer}\ansref{ex:sbox:safe}
% \begin{Verbatim}
%   Cryptol> :safe SubByte'
%   "SubByte'" is safe; no safety violations exist.
% \end{Verbatim}
% \end{Answer}

%=====================================================================
% \section{The {\ttfamily{\textbf ShiftRows}} transformation}
% \label{aes:shiftrows}
\sectionWithAnswers{The {\ttfamily{\textbf ShiftRows}} transformation}{aes:shiftrows}

\todo[inline]{Need a figure here to get reader in the right frame of
  mind.}

The second transformation AES\indAES utilizes is the {\tt
  ShiftRows}\indAESShiftRows operation~\cite[Section 5.1.2]{aes}. This
operation treats the {\tt State} as a $4\times4$ matrix, and rotates
the last three row to the left by the amounts 1, 2, and 3;
respectively.  Implementing {\tt ShiftRows} in Cryptol is trivial,
using the \verb+<<<+\indRotLeft operator:
\begin{code}
  ShiftRows : State -> State
  ShiftRows state = [ row <<< shiftAmount | row <- state
                                          | shiftAmount <- [0 .. 3]
                    ]
\end{code}

\begin{Exercise}\label{ex:aesshiftrows:0}
  Can you transform a state back into itself by repeated applications
  of {\tt ShiftRows}? How many times would you need to shift?  Verify
  your answer by writing and proving a corresponding Cryptol property.
\end{Exercise}
\begin{Answer}\ansref{ex:aesshiftrows:0}
  Consider what happens after 4 calls to {\tt ShiftRows}. The first
  row will stay the same, since it never moves. The second row moves
  one position each time, and hence it will move 4 positions at the
  end, restoring it back to its original starting
  configuration. Similarly, row 3 will rotate $2\times4=8$ times,
  again restoring it. Finally, row 4 rotates 3 times each for a total
  of $3\times4 = 12$ times, cycling back to its starting
  position. Hence, every 4th rotation will restore the entire state
  back. We can verify this in Cryptol by the following property:
\begin{code}
  shiftRow4RestoresBack : State -> Bit
  property shiftRow4RestoresBack state = states @ 4 == state
     where states = [state] # [ ShiftRows s | s <- states ]
\end{code}
We have:
\begin{Verbatim}
  Cryptol> :prove shiftRow4RestoresBack
  Q.E.D.
\end{Verbatim}
Of course, any multiple of 4 would have the same effect.
\end{Answer}

%=====================================================================
% \section{The {\ttfamily{\textbf MixColumns}} transformation}
% \label{sec:aesmixcolumns}
\sectionWithAnswers{The {\ttfamily{\textbf MixColumns}} transformation}{sec:aesmixcolumns}

The third major transformation AES\indAES performs is the {\tt
  MixColumns}\indAESMixColumns operation~\cite[Section 5.1.3]{aes}.
In this transformation, the {\tt State} is viewed as a $4\times4$
matrix, and each successive column of it is replaced by the result of
multiplying it by the matrix:
$$
\left[ \begin{array}{cccc} 2 & 3 & 1 & 1 \\ 1 & 2 & 3 & 1 \\ 1 & 1 & 2 & 3 \\ 3 & 1 & 1 & 2 \end{array} \right]
$$

As you might recall from linear algebra, given two \emph{compatible}
matrices $A$ and $B$, the $ij$th element of $A\times B$ is the
dot-product of the $i$th row of $A$ and the $j$th column of $B$.
(By \emph{compatible} we mean the number of columns of $A$ must be the
same as the number of rows of $B$.  All our matrices are $4\times 4$,
so they are always compatible.)  The dot-product is defined as
multiplying the corresponding elements of two same-length vectors and
adding the results together. The only difference here is that we use
the functions {\tt gf28Add} and {\tt gf28Mult} for addition and
multiplication respectively. We will develop this algorithm in the
following sequence of exercises.

\begin{Exercise}\label{ex:aesmc:0}
Write a function {\tt gf28DotProduct} with the signature:
\begin{code}
  gf28DotProduct : {n} (fin n) => ([n]GF28, [n]GF28) -> GF28
\end{code}
such that {\tt gf28DotProduct} returns the dot-product of two length $n$ vectors of GF($2^8$) elements.
\end{Exercise}
\begin{Answer}\ansref{ex:aesmc:0}
\begin{code}
  gf28DotProduct (xs, ys) = gf28Add [ gf28Mult (x, y) | x <- xs
                                                      | y <- ys
                                    ]
\end{code}
\end{Answer}

\todo[inline]{Check correct with theorem support.}

\begin{Exercise}\label{ex:aesmc:1}
  Write properties stating that the vector operation
  \texttt{gf28DotProduct} is commutative and distributive over vector
  addition:
\begin{eqnarray*}
a \cdot b &=& b \cdot a \\
a \cdot (b + c) &=& a\cdot b + a\cdot b
\end{eqnarray*}
Addition over vectors is defined element-wise. Prove the commutativity
property for vectors of length 10.  Distributivity will take much
longer, so you might want to do a \texttt{:check} on it.\indCmdCheck
\todo[inline]{Explain why distributivity is so much harder to prove.}
\end{Exercise}
\begin{Answer}\ansref{ex:aesmc:1}
\fvset{fontsize=\small}
\begin{code}
  property DPComm a b =     gf28DotProduct (a, b) == gf28DotProduct (b, a)
  property DPDist a b c =   gf28DotProduct (a, vectorAdd(b, c)) ==
                            gf28Add [ab, ac]
     where   ab = gf28DotProduct (a, b)
             ac = gf28DotProduct (a, c)
             vectorAdd (xs, ys) = [ gf28Add [x, y] | x <- xs
                                                   | y <- ys
                                  ]
\end{code}
We have:
\fvset{fontsize=\normalsize}
\begin{Verbatim}
  AES> :prove DPComm : [10]GF28 -> [10]GF28 -> Bit
  Q.E.D.
  AES> :check DPDist : [10]GF28 -> [10]GF28 -> [10]GF28 -> Bit
  Using random testing.
  passed 1000 tests.
\end{Verbatim}
\todo[inline]{Revisit after \ticket{207} is addressed.}
  % \verb+[Coverage: 0.00%. (1000/17668470647783843295832975007429185158274838968756189 ...58121606201292619776)]+
  % The coverage number on {\tt DPDist} is a clear indication on how
  % wild these property can get fairly quickly. (The total number of
  % cases is $2^{3*10*8} = 2^{240}$, which is a truly enormous number!)
You might be surprised that the total number of cases for this
property is $2^{3*10*8} = 2^{240}$---a truly ginormous number!
\end{Answer}

\begin{Exercise}\label{ex:aesmc:2}
Write a function
\begin{code}
  gf28VectorMult : {n, m, k} (fin n) => ([n]GF28, [m][n]GF28) ->
                                         [m]GF28
\end{code}
computing the dot-product of its first argument with each of the $m$
rows of the second argument, returning the resulting values as a
vector of $m$ elements.
\end{Exercise}
\begin{Answer}\ansref{ex:aesmc:2}
\begin{code}
  gf28VectorMult (v, ms) = [ gf28DotProduct(v, m) | m <- ms ]
\end{code}
\end{Answer}

\begin{Exercise}\label{ex:aesmc:3}
Write a function
\begin{code}
  gf28MatrixMult : {n, m, k} (fin m) => ([n][m]GF28, [m][k]GF28) 
                                      -> [n][k]GF28
\end{code}
which multiplies the given matrices in GF($2^8$)\indGF.
\end{Exercise}
\begin{Answer}\ansref{ex:aesmc:3}
  We simply need to call {\tt gfVectorMult} of the previous exercise
  on every row of the first matrix, after transposing the second
  matrix to make sure columns are properly aligned. We
  have:\indTranspose
\begin{code}
  gf28MatrixMult (xss, yss) = [ gf28VectorMult(xs, yss')
                              | xs <- xss ]
       where yss' = transpose yss
\end{code}
\end{Answer}

\todo[inline]{Remind the reader of the context again before jumping
  into \texttt{MixColumns}.}

\unparagraph Now that we have the matrix multiplication machinery
built, we can code {\tt MixColumns} fairly easily. Following the
description in the AES\indAES standard~\cite[Section 5.3.1]{aes}, all
we have to do is to multiply the matrix we have seen at the beginning
of this section with the state:

\begin{code}
  MixColumns : State -> State
  MixColumns state = gf28MatrixMult (m, state)
    where m = [[2, 3, 1, 1],
               [1, 2, 3, 1],
               [1, 1, 2, 3],
               [3, 1, 1, 2]]
\end{code}

Note that Cryptol makes no built-in assumption about row- or
column-ordering of multidimensional matrices.  Of course, given
Cryptol's concrete syntax, it makes little sense to do anything but
row-based ordering.

%=====================================================================
% \section{Key expansion}
% \label{aes:keyexpansion}
\sectionWithAnswers{Key expansion}{aes:keyexpansion}

\todo[inline]{Will we push the pipeline of verification all the way
  through?  In particular, I'm interested in the equivalence proofs
  between the various ticked and unticked versions of functions and
  their composition.}

Recall from Section~\ref{sec:aesparams} that AES takes 128, 192, or
256-bit keys. The key is not used as-is, however. Instead, AES\indAES
expands the key into a number of round keys, called the {\em key
  schedule}. Construction of the key schedule relies on a number of
auxiliary definitions, as we shall see shortly.

\paragraph*{Round constants} The AES\indAES standard refers to the constant
array {\tt Rcon} used during key expansion. For each {\tt i}, {\tt
  Rcon[i]} contains 4 words, the last three being 0~\cite[Section
5.2]{aes}.  The first element is given by $x^{i-1}$, where
exponentiation is done using the {\tt gf28Pow} function you have
defined in Exercise~\ref{aes:subbytes}-\ref{ex:gfmi:0}. In Cryptol,
it is easiest to define {\tt Rcon} as a function:
\begin{code}
  Rcon : [8] -> [4]GF28
  Rcon i = [(gf28Pow (<| x |>, i-1)), 0, 0, 0]
\end{code}

\begin{Exercise}\label{ex:aeskerc:0}
  By definition, AES\indAES only calls {\tt Rcon} with the parameters
  ranging from 1--10.  Based on this, create a table-lookup version
\begin{code}
  Rcon' : [8] -> [4]GF28
\end{code}
that simply performs a look-up instead. \lhint{Use Cryptol to find out
  what the elements of your table should be.}
\end{Exercise}
\begin{Answer}\ansref{ex:aeskerc:0}
Finding out the elements is easy:
\begin{Verbatim}
  Cryptol> [ (Rcon i)@0 | i <- [1 .. 10] ]
  [1, 2, 4, 8, 16, 32, 64, 128, 27, 54]
\end{Verbatim}
Note that we only capture the first element in each {\tt Rcon} value,
since we know that the rest are 0. We can now use this table to define
{\tt Rcon'} as follows:
\begin{code}
  Rcon' i = [(rcons @ (i-1)), 0, 0, 0]
      where rcons : [10]GF28
            rcons = [1, 2, 4, 8, 16, 32, 64, 128, 27, 54]
\end{code}
Note that we subtract 1 before indexing into the {\tt rcons} sequence
to get the indexing right.
\end{Answer}

\begin{Exercise}\label{ex:aeskerc:1}
  Write and prove a property that {\tt Rcon} and {\tt Rcon'} are
  equivalent when called with numbers in the range 1--10.
\end{Exercise}
\begin{Answer}\ansref{ex:aeskerc:1}
  We need to write a conditional property
  (Section~\ref{sec:condproof})\indThmCond. Below, we use the function
  {\tt elem} you have defined in
  Exercise~\ref{sec:recandrec}-\ref{ex:recfun:4:1}:\indElem
\begin{code}
  property RconCorrect x = if elem(x, [1..10])
                           then Rcon x == Rcon' x
                           else True
\end{code}
We have:
\begin{Verbatim}
  Cryptol> :prove RconCorrect
  Q.E.D.
\end{Verbatim}
\end{Answer}

\todo[inline]{Will we support the \texttt{:safe} command again? See
  \ticket{210}.  There is a commented-out exercise here on the safety
  of \texttt{Rcon'}.}
% \begin{Exercise}\label{ex:aeskerc:2}
% Is {\tt Rcon'} safe? Verify your answer using Cryptol.
% \end{Exercise}
% \begin{Answer}\ansref{ex:aeskerc:2}
% No, {\tt Rcon'} is {\em not} safe by itself:
% \begin{Verbatim}
%   Cryptol> :safe Rcon'
%   *** 1 safety condition to be checked.
%   *** Checking for violations of: ``..: index of symbolic-value is out of bounds
%   (valid range is 0 thru 9).''
%   *** Violation detected:
%   Rcon' 0x00
%            = ..: index of 255 is out of bounds
%            (valid range is 0 thru 9).
%   *** 1 problem found.
% \end{Verbatim}
% The premise of {\tt Rcon'} is that it must be called with numbers in
% the range 1--10. So, all its uses are actually safe, as we shall
% explore in Exercise~\ref{aes:keyexpansion}~\ref{ex:rconsafety}.
% \end{Answer}

\paragraph*{The {\ttfamily{\textbf SubWord}} function} The AES\indAES
specification refers to a function named {\tt SubWord}~\cite[Section
5.2]{aes}, that takes a 32-bit word and applies the {\tt SubByte}
transformation from Section~\ref{aes:subbytes}. This function is
trivial to code in Cryptol:
\begin{code}
  SubWord : [4]GF28 -> [4]GF28
  SubWord bs = [ SubByte' b | b <- bs ]
\end{code}
Note that we have used the table-lookup version ({\tt SubByte'},
Pg~\pageref{aes:subbytetl}) above.

\paragraph*{The {\ttfamily{\textbf RotWord}} function} The last
function we need for key expansion is named {\tt RotWord} by the
AES\indAES standard~\cite[Section 5.2]{aes}. This function merely
rotates a given word cyclically to the left:
\begin{code}
  RotWord : [4]GF28 -> [4]GF28
  RotWord [a0, a1, a2, a3] = [a1, a2, a3, a0]
\end{code}
We could have used \verb+<<<+\indRotLeft to implement {\tt RotWord} as
well, but the above definition textually looks exactly the one given
in the standard specification, and hence is preferable for the
purposes of clarity.

\todo[inline]{Demonstrate the use of rotate and prove the
  equivalence.}

\paragraph*{The key schedule} Recall that AES\indAES operates on 128,
192, or 256 bit keys.  These keys are used to construct a sequence of
so-called \emph{round keys}, each of which is 128 bits wide, and is
viewed the same way as the {\tt State}:\todo[inline]{\emph{Viewed} or
  \emph{used}?  Isn't this type abuse?}
\begin{code}
  type RoundKey = State
\end{code}
The expanded key schedule contains {\tt Nr}+1 round-keys.  (Recall
from Section~\ref{sec:aesparams} that {\tt Nr} is the number of
rounds.)  It also helps to separate out the first and the last keys,
as they are used in a slightly different fashion.  Based on this
discussion, we use the following Cryptol type to capture the
key schedule:
\begin{code}
  type KeySchedule = (RoundKey, [Nr-1]RoundKey, RoundKey) 
\end{code}

The key schedule is computed by seeding it with the initial key and
computing the successive elements from the previous entries.  In
particular, the $i$th element of the expanded key is determined as
follows, copied verbatim from the AES\indAES
specification~\cite[Figure 11; Section 5.2]{aes}:
\begin{Verbatim}
   temp = w[i-1]
   if (i mod Nk = 0)
       temp = SubWord(RotWord(temp)) xor Rcon[i/Nk]
   else if (Nk > 6 and i mod Nk = 4)
       temp = SubWord(temp)
   end if
   w[i] = w[i-Nk] xor temp
\end{Verbatim}
In the pseudo-code, the {\tt w} array contains the expanded key. We
are computing {\tt w[i]}, using the values {\tt w[i-1]} and {\tt
  w[i-Nk]}.  The result is the exclusive-or of {\tt w[i-Nk]} and a
mask value, called {\tt temp} above. The mask is computed using {\tt
  w[i-1]}, the {\tt Rcon} array we have seen before, the current index
{\tt i}, and {\tt Nk}.  This computation is best expressed as a
function in Cryptol that we will call {\tt NextWord}.  We will name
the {\tt w[i-1]} argument {\tt prev}, and the {\tt w[i-Nk]} argument
{\tt old}. Otherwise, the Cryptol code just follows the pseudo-code
above, written in a functional style to compute the mask:
\begin{code}
  NextWord : ([8],[4][8],[4][8]) -> [4][8]
  NextWord(i, prev, old) = old ^ mask
     where mask = if i % `Nk == 0
                  then SubWord(RotWord(prev)) ^ Rcon' (i / `Nk)
                  else if (`Nk > 6) && (i % `Nk == 4)
                       then SubWord(prev)
                       else prev
\end{code}
\note{It is well worth studying the pseudo-code above and the Cryptol
  equivalent to convince yourself they are expressing the same idea!}

To compute the key schedule we start with the initial key as the
seed. We then make calls to {\tt NextWord} with a sliding window of
{\tt Nk} elements, computing the subsequent elements. Let us first
write a function that will apply this algorithm to generate an
infinite regression of elements:\indDrop\indTranspose
\begin{code}
  ExpandKeyForever : [Nk][4][8] -> [inf]RoundKey
  ExpandKeyForever seed = [ transpose g | g <- groupBy`{4} keyWS ]
    where keyWS = seed # [ NextWord(i, prev, old)
                          | i    <- [`Nk ...]
                          | prev <- drop`{Nk-1} keyWS
                          | old  <- keyWS
                         ]
\end{code}
Note how {\tt prev} tracks the previous 32 bits of the expanded key
(by dropping the first {\tt Nk-1} elements), while {\tt old} tracks
the {\tt i-Nk}th recurrence for {\tt keyWS}. Once we have the
infinite expansion, it is easy to extract just the amount we need by
using number of rounds ({\tt Nr}) as our guide:\indIndex\indIndexs
\begin{code}
  ExpandKey : [AESKeySize] -> KeySchedule
  ExpandKey key = (keys @ 0, keys @@ [1 .. (Nr - 1)], keys @ `Nr)
    where  seed : [Nk][4][8]
           seed = split (split key)
           keys = ExpandKeyForever seed
\end{code}
The call {\tt split key} chops {\tt AESKey} into {\tt [Nk*4][8]}, and
the outer call to {\tt split} further constructs the {\tt [Nk][4][8]}
elements.

\paragraph*{Testing {\ttfamily{\textbf ExpandKey}}} The completion of
{\tt ExpandKey} is an important milestone in our AES\indAES
development, and it is worth testing it before we proceed. The AES
specification has example key expansions that we can use. The
following function will be handy in viewing the output correctly
aligned:
\begin{code}
  fromKS : KeySchedule -> [Nr+1][4][32]
  fromKS (f, ms, l) = [ formKeyWords (transpose k) 
                      | k <- [f] # ms # [l] 
                      ]
         where formKeyWords bbs = [ join bs | bs <- bbs ]
\end{code}
Here is the example from Appendix~A.1 of the AES\indAES
specification~\cite{aes}:
\begin{Verbatim}
  Cryptol> fromKS (ExpandKey 0x2b7e151628aed2a6abf7158809cf4f3c)
  [[0x2b7e1516, 0x28aed2a6, 0xabf71588, 0x09cf4f3c],
   [0xa0fafe17, 0x88542cb1, 0x23a33939, 0x2a6c7605],
   [0xf2c295f2, 0x7a96b943, 0x5935807a, 0x7359f67f],
   [0x3d80477d, 0x4716fe3e, 0x1e237e44, 0x6d7a883b],
   [0xef44a541, 0xa8525b7f, 0xb671253b, 0xdb0bad00],
   [0xd4d1c6f8, 0x7c839d87, 0xcaf2b8bc, 0x11f915bc],
   [0x6d88a37a, 0x110b3efd, 0xdbf98641, 0xca0093fd],
   [0x4e54f70e, 0x5f5fc9f3, 0x84a64fb2, 0x4ea6dc4f],
   [0xead27321, 0xb58dbad2, 0x312bf560, 0x7f8d292f],
   [0xac7766f3, 0x19fadc21, 0x28d12941, 0x575c006e],
   [0xd014f9a8, 0xc9ee2589, 0xe13f0cc8, 0xb6630ca6]]
\end{Verbatim}
As you can verify this output matches the last column of the table in
Appendix~A.1 of the reference specification for AES.\indAES

\todo[inline]{A theorem for safety follows that we can re-introduce
  after \ticket{210} is closed.}
% \begin{Exercise}\label{ex:rconsafety}
%   Let us revisit Exercise~\ref{aes:keyexpansion}~\ref{ex:aeskerc:2},
%   where we have seen that {\tt Rcon'} is not safe. However, the
%   promise was that it would only be called with numbers ranging
%   1--10, thus all its uses would be fine. Verify this claim by
%   proving {\tt ExpandKey} itself is safe. How about {\tt NextWord}?
%   Is it safe?
% \end{Exercise}
% \begin{Answer}\ansref{ex:rconsafety}
% Let us first check {\tt NextWord}. We have:
% \begin{Verbatim}
%   Cryptol> :safe NextWord
%   *** 1 safety condition to be checked.
%   *** Checking for violations of: ``..: index of symbolic-value is out of bounds
%   (valid range is 0 thru 9).''
%   *** Violation detected:
%   NextWord (0x00, [0x00 0x00 0x00 0x00], [0x00 0x00 0x00 0x00])
%            = ..: index of 255 is out of bounds
%   (valid range is 0 thru 9).
%   *** 1 problem found.
% \end{Verbatim}
% This is indeed to be expected, since {\tt NextWord} simply uses its
% {\tt i} argument in the call to {\tt Rcon' (i/Nk)}, and there is no
% guarantee that this {\tt i} will not be 0. However, the use from
% within {\tt ExpandKey} is perfectly fine:
% \begin{Verbatim}
%   Cryptol> :safe ExpandKey
%   "ExpandKey" is safe; no safety violations exist.
% \end{Verbatim}
% formally ensuring that the use of {\tt Rcon'} via {\tt NextWord} in
% {\tt ExpandKey} is indeed safe.
% \end{Answer}

%=====================================================================
\section{The {\ttfamily{\textbf AddRoundKey}} transformation}
\label{sec:ttfam-addr-transf}

{\tt AddRoundKey} is the simplest of all the transformations in
AES~\cite[Section 5.1.4]{aes}.\indAES It merely amounts to the
exclusive-or of the state and the current round key:\indXOr
\begin{code}
  AddRoundKey : (RoundKey, State) -> State
  AddRoundKey (rk, s) = rk ^ s
\end{code}
Notice that Cryptol's {\tt \Verb|^|} operator applies structurally to
arbitrary shapes, computing the exclusive-or element-wise.

%=====================================================================
% \section{AES encryption}
% \label{sec:aes:encryption}
\sectionWithAnswers{AES encryption}{sec:aes:encryption}

We now have all the necessary machinery to perform AES\indAES
encryption.

\paragraph*{AES rounds} As mentioned before, AES performs encryption in
rounds. Each round consists of performing {\tt SubBytes}
(Section~\ref{aes:subbytes}), {\tt ShiftRows}
(Section~\ref{aes:shiftrows}), and {\tt MixColumns}
(Section~\ref{sec:aesmixcolumns}). Before finishing up, each round
also adds the current round key to the state~\cite[Section
5.1]{aes}. The Cryptol code for the rounds is fairly trivial:
\begin{code}
  AESRound : (RoundKey, State) -> State
  AESRound (rk, s) = AddRoundKey (rk,
                                  MixColumns (ShiftRows (SubBytes s)))
\end{code}

\paragraph*{The final round} The last round of AES is slightly
different than the others. It omits the {\tt MixColumns}
transformation:
\begin{code}
  AESFinalRound : (RoundKey, State) -> State
  AESFinalRound (rk, s) = AddRoundKey (rk, ShiftRows (SubBytes s))
\end{code}

\paragraph*{Forming the input/output blocks} Recall that AES processes
input in blocks of 128 bits, producing 128 bits of output, regardless
of the key size.  We will need two helper functions to convert 128-bit
messages to and from AES states.  Conversion from a message to a state
is easy to define:
\begin{code}
  msgToState : [128] -> State
  msgToState msg = transpose (split (split msg))
\end{code}
The first call to \texttt{split} gives us four 32-bit words, which we again
split into bytes.  We then form the AES state by transposing the
resulting matrix.  In the other direction, we simply transpose the
state and perform the necessary
\texttt{join}s:\indTranspose\indJoin\indReverse\indSplit
\begin{code}
  stateToMsg : State -> [128]
  stateToMsg st = join (join (transpose st))
\end{code}

\begin{Exercise}\label{ex:aes:enc0}
  Write and prove a pair of properties stating that {\tt msgToState}
  and {\tt stateToMsg} are inverses of each other.
\end{Exercise}
\begin{Answer}\ansref{ex:aes:enc0}
\begin{code}
  property msgToStateToMsg msg = stateToMsg(msgToState(msg)) == msg
  property stToMsgToSt s = msgToState(stateToMsg s) == s
\end{code}
We have:
\begin{Verbatim}
  Cryptol> :prove msgToStateToMsg
  Q.E.D.
  Cryptol> :prove stToMsgToSt
  Q.E.D.
\end{Verbatim}
\end{Answer}

\paragraph*{Putting it together} To encrypt, AES merely expands the
given key and calls the round functions. The starting state (\texttt{state0}
below) is constructed by adding the first round key to the
input. We then run all the middle rounds using a simple comprehension,
and finish up by applying the last round~\cite[Figure 5, Section
5.1]{aes}:\indRIndex
\begin{code}
  aesEncrypt : ([128], [AESKeySize]) -> [128]
  aesEncrypt (pt, key) = stateToMsg (AESFinalRound (kFinal, rounds ! 0))
    where   (kInit, ks, kFinal) = ExpandKey key
            state0 = AddRoundKey(kInit, msgToState pt)
            rounds = [state0] # [ AESRound (rk, s) | rk <- ks
                                                   | s <- rounds
                                ]
\end{code}

\paragraph*{Testing} We can now run some test vectors. Note that, just
because a handful of test vectors pass, we cannot claim that our
implementation of AES is correct.

The first example comes from Appendix~B of the AES\indAES
standard~\cite{aes}:
\begin{Verbatim}
  Cryptol> aesEncrypt (0x3243f6a8885a308d313198a2e0370734,   \
                       0x2b7e151628aed2a6abf7158809cf4f3c)
  0x3925841d02dc09fbdc118597196a0b32
\end{Verbatim}
which is what the standard asserts to be the answer. (Note that you
have to read the final box in Appendix~B column-wise!)  The second
example comes from Appendix~C.1:
\begin{Verbatim}
  Cryptol> aesEncrypt (0x00112233445566778899aabbccddeeff,   \
                       0x000102030405060708090a0b0c0d0e0f)
  0x69c4e0d86a7b0430d8cdb78070b4c55a
\end{Verbatim}
Again, the result agrees with the standard.

\commentout{
\begin{code}
  aesEncSanityCheck =  (aesEncrypt(0x3243f6a8885a308d313198a2e0370734, 0x2b7e151628aed2a6abf7158809cf4f3c) == 0x3925841d02dc09fbdc118597196a0b32) &&  (aesEncrypt(0x00112233445566778899aabbccddeeff, 0x000102030405060708090a0b0c0d0e0f) == 0x69c4e0d86a7b0430d8cdb78070b4c55a)
\end{code}
}

\commentout{
192 and 256 bit tests:
\begin{Verbatim}
aesEncrypt(0x00112233445566778899aabbccddeeff, 0x000102030405060708090a0b0c0d0e0f1011121314151617) == 0xdda97ca4864cdfe06eaf70a0ec0d7191
aesEncrypt(0x00112233445566778899aabbccddeeff, 0x000102030405060708090a0b0c0d0e0f101112131415161718191a1b1c1d1e1f) == 0x8ea2b7ca516745bfeafc49904b496089
\end{Verbatim}
}

\paragraph*{Other key sizes} Our development of AES has been key-size
agnostic, relying on the definition of the parameter \texttt{Nk}.  (See
Section~\ref{sec:aesparams}.)  To obtain AES192, all we need is to set
\texttt{Nk} to be 6, no additional code change is needed.  Similarly, we
merely need to set \texttt{Nk} to be 8 for AES256.

\begin{Exercise}\label{ex:aes192}
  By setting \texttt{Nk} to be 6 and 8 respectively, try the test vectors
  given in Appendices C.2 and C.3 of the AES\indAES
  standard~\cite{aes}.
\end{Exercise}

%=====================================================================
% \section{Decryption}
% \label{sec:aesdecryption}
\sectionWithAnswers{Decryption}{sec:aesdecryption}

AES decryption is fairly similar to encryption, except it uses inverse
transformations~\cite[Figure 12, Section 5.3]{aes}. Armed with all the
machinery we have built so far, the inverse transformations are
relatively easy to define.

%~~~~~~~~~~~~~~~~~~~~~~~~~~~~~~~~~~~~~~~~~~~~~~~~~~~~~~~~~~~~~~~~~~~~~
\subsection{The {\ttfamily{\textbf InvSubBytes}} transformation}
\label{sec:ttfam-invs-transf}

The {\tt InvSubBytes} transformation reverses the {\tt SubBytes}
transformation of Section~\ref{aes:subbytes}. As with {\tt SubBytes},
we have a choice to either do a table lookup implementation, or follow
the mathematical description.  We will do the former in these examples;
you are welcome to do the latter on your own and prove the equivalence
of the two versions.  To do so, we need to invert the transformation
given by:
$$
 b \oplus b \ggg 4 \oplus b \ggg 5 \oplus b \ggg 6 \oplus b \ggg 7 \oplus c
$$
where $c$ is {\tt 0x63}. It turns out that the inverse of this
transformation can be computed by
$$
 b \ggg 2 \oplus b \ggg 5 \oplus b \ggg 7 \oplus d
$$
where $d$ is {\tt 0x05}.  It is easy to code this inverse transform in
Cryptol:

\begin{code}
  xformByte' : GF28 -> GF28
  xformByte' b = gf28Add [(b >>> 2), (b >>> 5), (b >>> 7), d]
    where d = 0x05
\end{code}

\begin{Exercise}\label{ex:invsb:1}
  Write and prove a Cryptol property stating that {\tt xformByte'} is
  the inverse of the function {\tt xformByte} that you have defined in
  Exercise~\ref{aes:subbytes}~\ref{ex:aessbytes:0}.
\end{Exercise}
\begin{Answer}\ansref{ex:invsb:1}
\begin{code}
  property xformByteInverse x = xformByte' (xformByte x) == x
\end{code}
We have:
\begin{Verbatim}
  Cryptol> :prove xformByteInverse
  Q.E.D.
\end{Verbatim}
\end{Answer}

\unparagraph We can now define the inverse S-box
transform,\indAESInvSbox using the multiplicative inverse function
{\tt gf28Inverse} you have defined in
Exercise~\ref{aes:subbytes}~\ref{ex:gfmi:01}:
\begin{code}
  InvSubByte : GF28 -> GF28
  InvSubByte b = gf28Inverse (xformByte' b)

  InvSubBytes : State -> State
  InvSubBytes state = [ [ InvSubByte b | b <- row ] 
                      | row <- state
                      ]
\end{code}

\begin{Exercise}\label{ex:invsb:2}
  Write and prove a Cryptol property showing that {\tt InvSubByte}
  reverses {\tt SubByte}.\indAESSbox
\end{Exercise}
\begin{Answer}\ansref{ex:invsb:2}
\begin{code}
  property sboxInverse s = InvSubBytes (SubBytes s) == s
\end{code}
We have:
\begin{Verbatim}
  Cryptol> :prove xformByteInverse
  Q.E.D.
\end{Verbatim}
\end{Answer}

\begin{Exercise}\label{ex:invsb:3}
  The AES specification provides an inverse S-box table~\cite[Figure
  14, Section 5.3.2]{aes}. Write a Cryptol function {\tt InvSubBytes'}
  using the table lookup technique. Make sure your implementation is
  correct (i.e., equivalent to {\tt InvSubBytes}) by writing and
  proving a corresponding property.
\end{Exercise}

\subsection{The {\ttfamily{\textbf InvShiftRows}} transformation}
\label{sec:ttfam-invsh-transf}

The {\tt InvShiftRows} transformation\indAESInvShiftRows simply
reverses the {\tt ShiftRows}\indAESShiftRows transformation from
Section~\ref{aes:shiftrows}:

\begin{code}
  InvShiftRows : State -> State
  InvShiftRows state = [ row >>> shiftAmount 
                       | row <- state
                       | shiftAmount <- [0 .. 3]
                       ]
\end{code}

\begin{Exercise}\label{ex:aesisr:0}
  Write and prove a property stating that {\tt InvShiftRows} is the
  inverse of {\tt ShiftRows}.
\end{Exercise}
\begin{Answer}\ansref{ex:aesisr:0}
\begin{code}
  property shiftRowsInverse s = InvShiftRows (ShiftRows s) == s
\end{code}
We have:
\begin{Verbatim}
  Cryptol> :prove shiftRowsInverse
  Q.E.D.
\end{Verbatim}
\end{Answer}

\subsection{The {\ttfamily{\textbf InvMixColumns}} transformation}
\label{sec:ttfam-invm-transf}

Recall from Section~\ref{sec:aesmixcolumns} that {\tt
  MixColumns}\indAESMixColumns amounts to matrix multiplication in
GF($2^8$).\indGF The inverse transform turns out to be the same,
except with a different matrix:\indAESInvMixColumns

\begin{code}
  InvMixColumns : State -> State
  InvMixColumns state = gf28MatrixMult (m, state)
    where m = [[0x0e, 0x0b, 0x0d, 0x09],
               [0x09, 0x0e, 0x0b, 0x0d],
               [0x0d, 0x09, 0x0e, 0x0b],
               [0x0b, 0x0d, 0x09, 0x0e]]
\end{code}

\begin{Exercise}\label{ex:aesimc:0}
  Write and prove a property stating that {\tt InvMixColumns} is the
  inverse of {\tt MixColumns}.
\end{Exercise}
\begin{Answer}\ansref{ex:aesimc:0}
\begin{code}
  property mixColumnsInverse s = InvMixColumns (MixColumns s) == s
\end{code}
Unlike others, this property is harder to prove automatically and will
take much longer. Below we show the {\tt :check} results instead:
\begin{Verbatim}
  Cryptol> :check mixColumnsInverse
  Checking case 1000 of 1000 (100.00%) 
  1000 tests passed OK
\end{Verbatim}

\todo[inline]{Reflect upon coverage, in relation to \ticket{207}.}
  % \verb+[Coverage: 0.00%. (1000/340282366920938463463374607431768211456)]+.
  % The coverage information is an indication of the magnitude of the
  % problem we are dealing with.}
\end{Answer}

%=====================================================================
\section{The inverse cipher}
\label{sec:inverse-cipher}

We now also have all the ingredients to encode AES decryption.
Following Figure~12 (Section 5.3) of the AES\indAES
standard~\cite{aes}: {\small
\begin{code}
  AESInvRound : (RoundKey, State) -> State
  AESInvRound (rk, s) =
    InvMixColumns (AddRoundKey (rk, InvSubBytes (InvShiftRows s)))

  AESFinalInvRound : (RoundKey, State) -> State
  AESFinalInvRound (rk, s) = AddRoundKey (rk, InvSubBytes (InvShiftRows s))

  aesDecrypt : ([128], [AESKeySize]) -> [128]
  aesDecrypt (ct, key) = stateToMsg (AESFinalInvRound (kFinal, rounds ! 0))
    where
      (kFinal, ks, kInit) = ExpandKey key
      state0 = AddRoundKey(kInit, msgToState ct)
      rounds = [state0] # [ AESInvRound (rk, s) 
                          | rk <- reverse ks
                          | s  <- rounds
                          ]
\end{code}
}

Note how we use the results of {\tt ExpandedKey}, by carefully naming
the first and last round keys and using the middle keys in reverse.

\paragraph*{Testing} Let us repeat the tests for AES encryption. Again,
the first example comes from Appendix~B of the AES\indAES
standard~\cite{aes}:
\begin{Verbatim}
  Cryptol> aesDecrypt (0x3925841d02dc09fbdc118597196a0b32,    \
                       0x2b7e151628aed2a6abf7158809cf4f3c)
  0x3243f6a8885a308d313198a2e0370734
\end{Verbatim}
which agrees with the original value. The second example comes from
Appendix~C.1:
\begin{Verbatim}
  Cryptol> aesDecrypt (0x69c4e0d86a7b0430d8cdb78070b4c55a,   \
                       0x000102030405060708090a0b0c0d0e0f)
  0x00112233445566778899aabbccddeeff
\end{Verbatim}
Again, the result agrees with the standard.

\commentout{
\begin{code}
  aesDecSanityCheck = (aesDecrypt(0x3925841d02dc09fbdc118597196a0b32,  0x2b7e151628aed2a6abf7158809cf4f3c) == 0x3243f6a8885a308d313198a2e0370734) && (aesDecrypt(0x69c4e0d86a7b0430d8cdb78070b4c55a,  0x000102030405060708090a0b0c0d0e0f) == 0x00112233445566778899aabbccddeeff)
\end{code}
}

\commentout{
192 and 256 bit tests:
\begin{Verbatim}
aesDecrypt(0xdda97ca4864cdfe06eaf70a0ec0d7191, 0x000102030405060708090a0b0c0d0e0f1011121314151617) == 0x00112233445566778899aabbccddeeff
aesDecrypt(0x8ea2b7ca516745bfeafc49904b496089, 0x000102030405060708090a0b0c0d0e0f101112131415161718191a1b1c1d1e1f) == 0x00112233445566778899aabbccddeeff
\end{Verbatim}
}

\paragraph*{Other key sizes} Similar to encryption, all we need to
obtain AES192 decryption is to set {\tt Nk} to be 6 in
Section~\ref{sec:aesparams}. Setting {\tt Nk} to 8 will
correspondingly give us AES256.

\paragraph*{The code} You can see all the Cryptol code for AES in
Appendix~\ref{app:aes}.

%=====================================================================
\section{Correctness}
\label{sec:aescorrectattempt}

While test vectors do provide good evidence of AES\indAES working
correctly, they do not provide a proof that we have implemented the
standard faithfully. In fact, for a block cipher like AES, it is not
possible to state what correctness would mean.  Tweaking some
parameters, or changing the S-box appropriately can give us a brand
new cipher.  And it would be impossible to tell this new cipher apart
from AES aside from running it against published test vectors.

\todo[inline]{Is this claim about correctness really true?!}

What we can do, however, is gain assurance that our implementation
demonstrably has the desired properties. We have done this throughout
this chapter by stating and proving a number of properties about AES
and its constituent parts. The Cryptol methodology allows us to
construct the code together with expected properties, allowing
high-assurance development. We conclude this chapter with one final
property, stating that {\tt aesEncrypt} and {\tt aesDecrypt} do indeed
form an encryption-decryption pair:

\begin{code}
  property AESCorrect msg key = aesDecrypt (aesEncrypt (msg, key), key) == msg
\end{code}

Can we hope to automatically prove this theorem?  For 128-bit AES, the
state space for this theorem has $2^{256}$ elements.  It would be
naive to expect that we can prove this theorem by a push-button tool
very quickly.\footnote{Note that, for a general algorithm with this
  large a state space, it is entirely possible to perform automatic
  verification using modern solvers, but if one carefully reflects
  upon the nature of cryptographic functions, it becomes clear why it
  should \emph{not} be the case here.}  We can however, gain some
assurance by running it through the {\tt :check} command:\indCmdCheck

\begin{Verbatim}
  Checking case 1000 of 1000 (100.00%) 
  1000 tests passed OK
\end{Verbatim}
\todo[inline]{The \texttt{:check} output here is out of date.}
%  [Coverage: 0.00%. (1000/11579208923731619542357098500868790785326998466564
%                                          ...0564039457584007913129639936)]
You will notice that even running quick-check will take a while for
the above theorem, and the total state space for this function means
that we have not even scratched the surface! That said, being able to
specify these properties together with very high level code is what
distinguishes Cryptol from other languages when it comes to
cryptographic algorithm programming.

\todo[inline]{This is highly dissatisfying!}

\todo[inline]{Add some text here back when we improve proof
  capabilities?}
%% Furthermore, Cryptol has some further proof tools in its arsenal
%% that will actually let us complete this proof in a reasonable
%% amount of time, as we shall see in Section~\ref{sec:proveaes}.
\commentout{
\begin{code}
  sanityCheck = (aesEncrypt (0x3243f6a8885a308d313198a2e0370734, 0x2b7e151628aed2a6abf7158809cf4f3c) 
                            == 0x3925841d02dc09fbdc118597196a0b32)
                        &&
                (aesEncrypt (0x00112233445566778899aabbccddeeff, 0x000102030405060708090a0b0c0d0e0f)
                            == 0x69c4e0d86a7b0430d8cdb78070b4c55a)
\end{code}
}

\todo[inline]{Need a punchier conclusion to the chapter, since it ends
  the book, or add a new conclusion.}

%%% Local Variables: 
%%% mode: latex
%%% TeX-master: "../main/Cryptol"
%%% End: 
