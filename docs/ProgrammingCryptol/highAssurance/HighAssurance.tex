\commentout{
\begin{code}
  module HighAssurance where

  import Classic
  import Enigma
\end{code}
}

%#####################################################################
\chapter{High-assurance programming}
\label{cha:high-assur-progr}

Writing correct software is the holy grail of programming. Bugs
inevitably exist, however, even in thoroughly tested projects.  One
fundamental issue is the lack of support in typical programming
languages to let the programmer \emph{state} what it means to be
correct, let alone formally establish any notion of correctness.  To
address this shortcoming, Cryptol advocates the high-assurance
programming approach: programmers explicitly state correctness
properties\indProperty along with their code, which are explicitly
checked by the Cryptol toolset.  Properties are not comments or mere
annotations, so there is no concern that they will become obsolete as
your code evolves.  The goal of this chapter is to introduce you to
these tools, and to the notion of high-assurance programming in
Cryptol via examples.

%=====================================================================
% \section{Writing properties}
% \label{sec:writingproperties}
\sectionWithAnswers{Writing properties}{sec:writingproperties}

Consider the equality:
$$
x^2 - y^2 = (x-y) * (x+y)
$$
Let us write two Cryptol functions that capture both sides of this
equation:\indTimes\indExponentiate\indMinus\indPlus
\begin{code}
  sqDiff1 (x, y) = x^^2  - y^^2
  sqDiff2 (x, y) = (x-y) * (x+y)
\end{code}
We would like to express the property that {\tt sqDiff1} and {\tt
  sqDiff2} are precisely the same functions: Given the same {\tt x}
and {\tt y}, they should return exactly the same answer. We can
express this property in Cryptol using a {\tt properties}
declaration:\indProperty
\begin{code}
  sqDiffsCorrect : ([8], [8]) -> Bit
  property sqDiffsCorrect (x, y) = sqDiff1 (x, y) == sqDiff2 (x, y)
\end{code}
The above declaration reads as follows: {\tt sqDiffsCorrect} is a
property stating that for all {\tt x} and {\tt y}, the expression {\tt
  sqDiff1 (x, y) == sqDiff2(x, y)} evaluates to {\tt
  True}. Furthermore, the type signature restricts the type of the
property to apply to only 8-bit values. As usual, the type signature
is optional.\indSignature If not given, Cryptol will infer one for
you.

\note{It is important to emphasize that the mathematical equality
  above and the Cryptol property are \emph{not} stating precisely the
  same property.  Remember that all Cryptol arithmetic is
  modular,\indModular while the mathematical equation is over
  arbitrary numbers, including negative, real, or even complex
  numbers.  The takeaway of this discussion is that we are only using
  this example for illustration purposes: Cryptol properties relate to
  Cryptol programs, and should not be used for expressing mathematical
  theorems (unless, of course, you are stating group theory theorems
  or theorems in an appropriate algebra)!  In particular, {\tt
    sqDiffsCorrect} is a property about the Cryptol functions {\tt
    sqDiff1} and {\tt sqDiff2}, not about the mathematical equation
  that inspired it.}

\begin{Exercise}\label{ex:thm:0}
  Write a property {\tt revRev} stating that {\tt reverse} of a {\tt
    reverse} returns a sequence unchanged.\indReverse
\end{Exercise}
\begin{Answer}\ansref{ex:thm:0}\indReverse
\begin{code}
  property revRev xs = reverse (reverse xs) == xs
\end{code}
\end{Answer}

\begin{Exercise}\label{ex:thm:1}
  Write a property {\tt appAssoc} stating that append is an
  associative operator.\indAppend
\end{Exercise}
\begin{Answer}\ansref{ex:thm:1}\indAppend
\begin{code}
  property appAssoc (xs, ys, zs) = xs # (ys # zs) == (xs # ys) # zs
\end{code}
\end{Answer}

\begin{Exercise}\label{ex:thm:2}
  Write a property {\tt revApp} stating that appending two sequences
  and reversing the result is the same as reversing the sequences and
  appending them in the reverse order, as illustrated in the following
  expression:\indReverse\indAppend
\begin{Verbatim}
   reverse ("HELLO" # "WORLD") == reverse "WORLD" # reverse "HELLO"
\end{Verbatim}
\end{Exercise}
\begin{Answer}\ansref{ex:thm:2}\indReverse\indAppend
\begin{code}
  property revApp (xs, ys) = reverse (xs # ys)
                             == reverse ys # reverse xs
\end{code}
\end{Answer}

\begin{Exercise}\label{ex:thm:3}
  Write a property {\tt lshMul} stating that shifting left by $k$ is
  the same as multiplying by $2^k$.
\end{Exercise}
\begin{Answer}\ansref{ex:thm:3}
\begin{code}
  property lshMul (n, k) = n << k == n * 2^^k
\end{code}
\end{Answer}

\note{A \texttt{property} declaration simply introduces a property about
  your program, which may or may \emph{not} actually hold. It is an
  assertion about your program, without any claim of correctness. In
  particular, you can clearly write properties that simply do not
  hold:}
\begin{code}
  property bogus x = x != x
\end{code}
It is important to distinguish between \emph{stating} a property and
actually \emph{proving} it. So far, our focus is purely on
specification. We will focus on actual proofs in
\autoref{sec:establishcorrectness}.

%~~~~~~~~~~~~~~~~~~~~~~~~~~~~~~~~~~~~~~~~~~~~~~~~~~~~~~~~~~~~~~~~~~~~~
\subsection{Property--function correspondence}\indThmFuncCorr
\label{sec:prop-funct-corr}

In Cryptol, properties can be used just like ordinary definitions:
\begin{Verbatim}
  Cryptol> sqDiffsCorrect (3, 5)
  True
  Cryptol> :t sqDiffsCorrect
  sqDiffsCorrect : ([8],[8]) -> Bit
\end{Verbatim}
That is, a property over {\tt$(x, y)$} is the same as a function over
the tuple {\tt (x, y)}. We call this the property--function
correspondence. Property declarations, aside from the slightly
different syntax, are \emph{precisely} the same as Cryptol functions
whose return type is \texttt{Bit}. There is no separate language for
writing or working with properties. We simply use the full Cryptol
language write both the programs and the properties that they satisfy.

%~~~~~~~~~~~~~~~~~~~~~~~~~~~~~~~~~~~~~~~~~~~~~~~~~~~~~~~~~~~~~~~~~~~~~
\subsection{Capturing test vectors}
\label{sec:thmvec}

One nice application of Cryptol properties is in capturing test
vectors:\indZero
\begin{code}
  property inctest = [ f x == y | (x, y) <- testVector ] == ~zero
    where  f x = x + 1
           testVector = [(3, 4), (4, 5), (12, 13), (78, 79)]
\end{code}
Notice that the property \texttt{inctest} does not have any parameters
(no \emph{forall} section), and thus acts as a simple \texttt{Bit} value
that will be true precisely when the given test case succeeds.

\todo[inline]{Figure out how to re-run Cryptol in this chapter to make
  sure the tweak to the above example (removal of dummy single formal
  parameter) works.}

%~~~~~~~~~~~~~~~~~~~~~~~~~~~~~~~~~~~~~~~~~~~~~~~~~~~~~~~~~~~~~~~~~~~~~
\subsection{Polymorphic properties}
\label{sec:polythm}

Just like functions, Cryptol properties can be polymorphic as well. If
you want to write a property for a polymorphic function, for instance,
your properties will naturally be polymorphic too. Here is a simple
trivial example:
\begin{code}
  property multShift x = x * 2 == x << 1
\end{code}
If we ask Cryptol the type of {\tt multShift}, we get:
\begin{Verbatim}
  Cryptol> :t multShift
  multShift : {a} (fin a, a >= 2) => [a] -> Bit
\end{Verbatim}
That is, it is a property about all words of size at least two. The
question is whether this property does indeed hold? In the particular
case of {\tt multShift} that is indeed the case, below are some
examples using the property--function correspondence:\indThmFuncCorr
\begin{Verbatim}
  Cryptol> multShift (5 : [8])
  True
  Cryptol> multShift (5 : [10])
  True
  Cryptol> multShift (5 : [16])
  True
\end{Verbatim}
However, this is \emph{not} always the case for all polymorphic Cryptol
properties! The following example demonstrates:
\begin{code}
  property flipNeverIdentity x = x != ~x
\end{code}
The property {\tt flipNeverIdentity} states that
complementing\indComplement the bits of a value will always result in
a different value: a property we might expect to hold
intuitively. Here is the type of {\tt flipNeverIdentity}:
\begin{Verbatim}
  Cryptol> :t flipNeverIdentity
  flip : {a} (fin a) => a -> Bit
\end{Verbatim}
So, the only requirement on {\tt flipNeverIdentity} is that it
receives some finite type.\indFin Let us try some examples:
\begin{Verbatim}
  Cryptol> flipNeverIdentity True
  True
  Cryptol> flipNeverIdentity 3
  True
  Cryptol> flipNeverIdentity [1 2]
  True
\end{Verbatim}
However:
\begin{Verbatim}
  Cryptol> flipNeverIdentity (0 : [0])
  False
\end{Verbatim}
That is, when given a {\tt 0}-bit wide value, the complement will in
fact do nothing and return its argument unchanged! Therefore, the
property {\tt flipNeverIdentity} is not valid, since it holds at
certain monomorphic types, but not at all types.\indMonomorphism

\begin{Exercise}\label{ex:polythm:1}
  Demonstrate another monomorphic type where {\tt flipNeverIdentity}
  does \emph{not} hold.
\end{Exercise}
\begin{Answer}\ansref{ex:polythm:1}
  There are many such types, all sharing the property that they do not
  take any space to represent. Here are a couple examples:\indZero
\begin{Verbatim}
  Cryptol> flipNeverIdentity (zero : ([0], [0]))
  False
  Cryptol> flipNeverIdentity (zero : [0][8])
  False
\end{Verbatim}
\end{Answer}

\note{The moral of this discussion is that the notion of polymorphic
  validity\indThmPolyvalid (i.e., that a given polymorphic property
  will either hold at all of its monomorphic instances or none) does
  not hold in Cryptol. A polymorphic property can be valid at some,
  all, or no instances of it.}

\begin{Exercise}\label{ex:polythm:2}
  The previous exercise might lead you to think that it is the 0-bit
  word type ({\tt [0]}) that is at the root of the polymorphic
  validity issue.  This is not true. Consider the following
  example:\indWidth\indThmPolyvalid
\begin{code}
  property widthPoly x = (w == 15) || (w == 531)
     where w = width x
\end{code}
What is the type of {\tt widthPoly}? At what instances does it hold?
Write a property {\tt evenWidth} that holds only at even-width word
instances.
\end{Exercise}
\begin{Answer}\ansref{ex:polythm:2}
\begin{Verbatim}
  Cryptol> :t widthPoly
  widthPoly : {a, b} (fin a) => [a]b -> Bit
\end{Verbatim}
It is easy to see that {\tt widthPoly} holds at the instances:
\begin{Verbatim}
  {b} [15]b -> Bit
\end{Verbatim}
and
\begin{Verbatim}
  {b} [531]b -> Bit
\end{Verbatim}
but at no other. Based on this, we can write {\tt evenWidth} as
follows:\indWidth
\begin{code}
  property evenWidth x = (width x) ! 0 == False
\end{code}
remembering that the 0'th bit of an even number is always {\tt
  False}. We have:
\begin{Verbatim}
  Cryptol> evenWidth (0:[1])
  False
  Cryptol> evenWidth (0:[2])
  True
  Cryptol> evenWidth (0:[3])
  False
  Cryptol> evenWidth (0:[4])
  True
  Cryptol> evenWidth (0:[5])
  False
\end{Verbatim}
\end{Answer}

%=====================================================================
% \section{Establishing correctness}
% \label{sec:establishcorrectness}
\sectionWithAnswers{Establishing correctness}{sec:establishcorrectness}

Our focus so far has been using Cryptol to \emph{state} properties of
our programs, without actually trying to prove them correct. This
separation of concerns is essential for a pragmatic development
approach.  Properties act as contracts that programmers state along
with their code, which can be separately checked by the
toolset~\cite{erkok-matthews-cryptolEqChecking-09}.  This approach
allows you to state the properties you want, and then work on your
code until the properties are indeed satisfied. Furthermore,
properties stay with your program forever, so they can be checked at a
later time to ensure changes (improvements/additions/optimizations
etc.) did not violate the stated properties.

%~~~~~~~~~~~~~~~~~~~~~~~~~~~~~~~~~~~~~~~~~~~~~~~~~~~~~~~~~~~~~~~~~~~~~
\subsection{Formal proofs}
\label{sec:formal-proofs}

Recall our very first property, {\tt sqDiffsCorrect}, from
\autoref{sec:writingproperties}. We will now use Cryptol to
actually prove it automatically.  To prove {\tt sqDiffsCorrect}, use
the command {\tt :prove}:\indCmdProve
\begin{Verbatim}
  Cryptol> :prove sqDiffsCorrect
  Q.E.D.
\end{Verbatim}
Note that the above might take a while to complete, as a formal proof
is being produced behind the scenes.  Once Cryptol formally
establishes the property holds, it prints ``{\tt Q.E.D.}'' to tell the
user the proof is complete.\indQED\indProve

\note{Cryptol uses off-the-shelf SAT\glosSAT and SMT\glosSMT solvers to
  perform these formal
  proofs~\cite{erkok-matthews-cryptolEqChecking-09}.  By default,
  Cryptol will use Microsoft Research's Z3 SMT solver under the hood,
  but it can be configured to use other SAT/SMT solvers as well, such
  as SRI's Yices~\cite{YicesWWW}\indYices, or CVC4
  ~\cite{cvc4WWW}.\footnote{To do this, first install the package(s)
    from the URLs provided in the bibliography. Once a prover has been
    installed you can activate it with, for example, {\tt :set
      prover=cvc4}.} Note that the {\tt :prove} command is a
  push-button tool: once the proof starts there is no user
  involvement. Of course, the external tool used may not be able to
  complete all the proofs in a feasible amount of time, naturally.

  \todo[inline]{Is this going to be a SAW-only capability, or Cryptol
    as well?  Ensure that a ticket exists for forward-porting this
    functionality.}

  % While it is out of the scope of this book, let us also mention that
  % Cryptol can also translate properties to Isabelle/HOL\indIsabelleHOL
  % for semi-automated theorem proving purposes~\cite{Isabelle-book},
  % where more complicated properties can be tackled with the human
  % guidance if necessary.
}

%~~~~~~~~~~~~~~~~~~~~~~~~~~~~~~~~~~~~~~~~~~~~~~~~~~~~~~~~~~~~~~~~~~~~~
\subsection{Counterexamples}
\label{sec:counterexamples}

Of course, properties can very well be invalid, due to bugs in code or
the specifications themselves. In these cases, Cryptol will always
print a counterexample value demonstrating why the property does not
hold. Here is an example demonstrating what happens when things go
wrong:
\begin{code}
  failure : [8] -> Bit
  property failure x = x == x+1
\end{code}
We have:
\begin{Verbatim}
  Cryptol> :prove failure
  failure 0 = False
\end{Verbatim}
Cryptol tells us that the property is falsifiable, and then
demonstrates a particular value ({\tt 0} in this case) that it fails
at. These counterexamples are extremely valuable for debugging
purposes.\indCounterExample

If you try to prove an invalid property that encodes a test vector
(\autoref{sec:thmvec}), then you will get a mere indication that
you have a contradiction, since there are no universally quantified
variables to instantiate to show you a
counterexample.\indContradiction  If the expression evaluates to {\tt
  True}, then it will be a trivial proof, as expected:
\begin{Verbatim}
  Cryptol> :prove False
  False = False
  Cryptol> :prove True
  Q.E.D.
  Cryptol> :prove 2 == 3
  (2 == 3) = False
  Cryptol> :prove reverse [1, 2] == [1, 2]
  (reverse [1, 2] == [1,2]) = False
  Cryptol> :prove 1+1 == 0
  Q.E.D.
\end{Verbatim}
The very last example demonstrates modular arithmetic in operation, as
usual.\indModular

%~~~~~~~~~~~~~~~~~~~~~~~~~~~~~~~~~~~~~~~~~~~~~~~~~~~~~~~~~~~~~~~~~~~~~
\subsection{Dealing with polymorphism}
\label{sec:deal-with-polym}

As we mentioned before, Cryptol properties can be polymorphic. As we
explored before, we cannot directly prove polymorphic properties as
they may hold for certain monomorphic instances while not for others.
In this cases, we must tell Cryptol what particular monomorphic
instance of the property we would like it to prove.  Let us
demonstrate this with the \texttt{multShift} property from
\autoref{sec:polythm}:
\begin{Verbatim}
  Cryptol> :prove multShift
    Not a monomorphic type:
    {a} (a >= 2, fin a) => [a] -> Bit
\end{Verbatim}
Cryptol is telling us that it cannot prove a polymorphic property
directly. We can, however, give a type annotation to monomorphise
it,\indTypeAnnotation and then prove it at a desired instance:
\begin{Verbatim}
  Cryptol> :prove multShift : [16] -> Bit
  Q.E.D.
\end{Verbatim}
In fact, you can use this very same technique to pass any bit-valued
function to the {\tt :prove} command:\indCmdProve\indWhere
\begin{Verbatim}
  Cryptol> :prove dbl where dbl x = (x:[8]) * 2 == x+x
  Q.E.D.
\end{Verbatim}
Of course, a \lamex (\autoref{sec:lamex}) would work just as well
too:\indLamExp
\begin{Verbatim}
  Cryptol> :prove \x -> (x:[8]) * 2 == x+x
  Q.E.D.
\end{Verbatim}

\begin{Exercise}\label{ex:prove:1}
  Prove the property {\tt revRev} you wrote in
  \exerciseref{ex:thm:0}. Try different
  monomorphic instantiations.
\end{Exercise}
\begin{Answer}\ansref{ex:prove:1}
  If we try to prove {\tt revRev} directly, we will get an error from
  Cryptol:
\begin{Verbatim}
  Cryptol> :prove revRev
  Not a valid predicate type:
  {a, b} (fin a, Cmp b) => [a]b -> Bit
\end{Verbatim}
Cryptol is telling us that the property has a polymorphic type, and
hence cannot be proven. We can easily prove instances of it, by either
creating new properties with fixed type signatures\indSignature, or by
monomorphising it via type annotations\indTypeAnnotation. Several
examples are given below:
\begin{Verbatim}
  Cryptol> :prove revRev : [10][8] -> Bit
  Q.E.D.
  Cryptol> :prove revRev : [100][32] -> Bit
  Q.E.D.
  Cryptol> :prove revRev : [0][4] -> Bit
  Q.E.D.
\end{Verbatim}
\end{Answer}

\begin{Exercise}\label{ex:prove:2}
  Prove the property {\tt appAssoc} you wrote in
  \exerciseref{ex:thm:1}, at several
  different monomorphic instances.
\end{Exercise}

\begin{Exercise}\label{ex:prove:3}
  Prove the property {\tt revApp} you wrote in
  \exerciseref{ex:thm:2}, at several
  different monomorphic instances.
\end{Exercise}

\begin{Exercise}\label{ex:prove:4}
  Prove the property {\tt lshMul} you wrote in
  \exerciseref{ex:thm:3}, at several
  different monomorphic instances.
\end{Exercise}

\begin{Exercise}\label{ex:prove:5}
  Use the {\tt :prove} command to prove and demonstrate
  counterexamples for the property {\tt widthPoly} defined in
  \exerciseref{ex:polythm:2}, using
  appropriate monomorphic instances.
\end{Exercise}
\begin{Answer}\ansref{ex:prove:5}
We have:
\begin{Verbatim}
  Cryptol> :prove widthPoly : [15] -> Bit
  Q.E.D.
  Cryptol> :prove widthPoly : [531] -> Bit
  Q.E.D.
  Cryptol> :prove widthPoly : [8] -> Bit
  widthPoly:[8] -> Bit 0 = False
  Cryptol> :prove widthPoly : [32] -> Bit
  widthPoly:[32] -> Bit 0 = False
\end{Verbatim}
\end{Answer}

%~~~~~~~~~~~~~~~~~~~~~~~~~~~~~~~~~~~~~~~~~~~~~~~~~~~~~~~~~~~~~~~~~~~~~
\subsection{Conditional proofs}
\label{sec:condproof}

It is often the case that we are interested in a property that only
holds under certain conditions.  For instance, in
\exerciseref{ex:arith:5:1} we have explored the
relationship between Cryptol's division, multiplication, and modulus
operators, where we asserted the following
property:\indMod\indDiv\indTimes
$$
  x = (x / y) \times y + (x\,\%\,y)
$$
Obviously, this relationship holds only when $y \not= 0$. The idea
behind a conditional Cryptol property\indThmCond is that we would like
to capture these side-conditions formally in our specifications.

We can use ordinary {\tt if-then-else} expressions in Cryptol to
write conditional properties.  If the condition is invalid, we simply return
{\tt True}, indicating that we are not interested in that particular
case. Depending on how natural it is to express the side-condition or
its negation, you can use one of the following two patterns:
% the following looks ODDLY laid out in the source but comes out OK when latex'ed
\begin{Verbatim}[commandchars=\\\{\}]
   if   \emph{side-condition-holds}         if    \emph{side-condition-fails}
   then \emph{property-expression}          then  True
   else True                         else  \emph{property-expression}
\end{Verbatim}

Alternatively, Cryptol provides a few logical connectives
(\texttt{==>}, \texttt{\textbackslash/}, and \texttt{/\textbackslash})
that are useful for writing conditional properties. Each is defined in
terms of \texttt{if-then-else} as follows:

\begin{Verbatim}
   a ==> b  =  if a then b else True
   a \/ b   =  if a then True else b
   a /\ b   =  if a then b else False
\end{Verbatim}

\begin{Exercise}\label{ex:cond:1}
  Express the relationship between division, multiplication, and
  modulus using a conditional Cryptol property.  Prove the property
  for various monomorphic instances.\indMod\indDiv\indTimes\indThmCond
\end{Exercise}
\begin{Answer}\ansref{ex:cond:1}
\begin{code}
  property divModMul (x,y) = if y == 0
                             then True   // precondition fails => True
                             else x == (x / y) * y + x % y
\end{code}
We have:
\begin{Verbatim}
  Cryptol> :prove divModMul : ([4], [4]) -> Bit
  Q.E.D.
  Cryptol> :prove divModMul : ([8], [8]) -> Bit
  Q.E.D.
\end{Verbatim}
\end{Answer}

%% TODO: Pedagogy here without exceptions
%% \note{It is tempting to write a function:}
%% \begin{code}
%%  implies (a, b) = if a then b else True
%%\end{code}
%%to simplify writing conditional properties.
%% However, this approach is flawed as it fails to account correctly for
%% exceptions. We shall revisit this topic in detail in
%% sections~\ref{sec:pitfall:evorder} and~\ref{sec:pitfall:thmexceptions}.

\paragraph*{Recognizing messages} Our work on classic ciphers
(\autoref{chapter:classic}) and the enigma
(\autoref{chapter:enigma}) involved working with messages that
contained the letters {\tt 'A'} .. {\tt 'Z'} only. When writing
properties about these ciphers it will be handy to have a recognizer
for such messages, as we explore in the next exercise.

\begin{Exercise}\label{ex:cond:2}
Write a function:
\begin{code}
  validMessage : {n} (fin n) => String n -> Bit
\end{code}
that returns {\tt True} exactly when the input only consists of the
letters {\tt 'A'} through {\tt 'Z'}. \lhint{Use the functions {\tt
    all} defined in \exerciseref{ex:zero:1},\indAll
  and {\tt elem} defined in
  \exerciseref{ex:recfun:4:1}.\indElem}
\end{Exercise}

\begin{Answer}\ansref{ex:cond:2}
  Using {\tt all}\indAll and {\tt elem}\indElem, it is easy to express
  {\tt validMessage}:
\begin{code}
  validMessage = all (\c -> elem (c, ['A' .. 'Z']))
\end{code}
Note the use of a \lamex to pass the function to {\tt all}. Of course,
we could have defined a separate function for it in a {\tt
  where}-clause\indWhere, but the function is short enough to directly
write it inline.
\end{Answer}

\todo[inline]{Should we revisit directly some past examples or
  solutions in the appendix and show what kinds of properties are
  appropriate to write and which ones can be verified?}

\begin{Exercise}\label{ex:cond:3}
  Recall the pair of functions {\tt caesar} and {\tt dCaesar} from
  \autoref{sec:caesar}.\indCaesarscipher Write a property, named
  {\tt caesarCorrect}, stating that {\tt caesar} and {\tt dCaesar} are
  inverses of each other for all {\tt d} (shift amount) and {\tt msg}
  (message)'s. Is your property valid? What extra condition do you
  need to assert on {\tt msg} for your property to hold? Prove the
  property for all messages of length 10.
\end{Exercise}
\begin{Answer}\ansref{ex:cond:3}
  A naive attempt would be to write:
\begin{code}
  property caesarCorrectBOGUS (d,msg) =
           dCaesar(d, caesar(d, msg)) == msg
\end{code}
However, this property is not correct for all {\tt msg}'s, since
Caesar's cipher only works for messages containing the letters {\tt
  'A'} \ldots {\tt 'Z'}, not arbitrary 8-bit values as the above
property suggests. We can see this easily by providing a bad input:
\begin{Verbatim}
  Cryptol> caesar (3, "1")
  invalid sequence index: 240
\end{Verbatim}
(240 is the difference between the ASCII value of {\tt '1'}, 49, and
the letter {\tt 'A'}, 65, interpreted as an 8-bit offset.) We should
use the {\tt validMessage} function of the previous exercise to write
a conditional property instead:
\begin{code}
  property caesarCorrect (d,msg) = if validMessage msg
                                   then dCaesar(d, caesar(d, msg)) == msg
                                   else True
\end{code}
We have:
\begin{Verbatim}
  Cryptol> :prove caesarCorrect : ([8], String(10)) -> Bit
  Q.E.D.
\end{Verbatim}
\end{Answer}

\begin{Exercise}\label{ex:cond:4}
  Write and prove a property for the {\tt modelEnigma} machine
  (page~\pageref{def:modelEnigma}), relating the {\tt enigma} and {\tt
    dEnigma} functions from \autoref{enigma:encdec}.
\end{Exercise}
\begin{Answer}\ansref{ex:cond:4}
\begin{code}
  property modelEnigmaCorrect pt =
       if validMessage pt
       then dEnigma (modelEnigma, enigma (modelEnigma, pt)) == pt
       else True
\end{code}
We have:
\begin{Verbatim}
  Cryptol> :prove modelEnigmaCorrect : String(10) -> Bit
  Q.E.D.
\end{Verbatim}
\end{Answer}
This may take a long time to prove, depending on the speed of your
machine, and the prover you choose.

%=====================================================================
% \section{Automated random testing}
% \label{sec:quickcheck}
\sectionWithAnswers{Automated random testing}{sec:quickcheck}

Cryptol's {\tt :prove} command\indCmdProve constructs rigorous formal
proofs using push-button tools.\footnote{While some of the solvers
  that Cryptol uses are capable of \emph{emitting} proofs, such
  functionality is not exposed as a Cryptol feature.}  The underlying
technique used by Cryptol (SAT-\glosSAT and SMT-based\glosSMT
equivalence checking) is complete\indProofCompleteness, i.e., it will
always either prove the property or find a
counterexample.\indCounterExample In the worst case, however, the
proof process might take infeasible amounts of resources, potentially
running out of memory or taking longer than the amount of time you are
willing to wait.

What is needed for daily development tasks is a mechanism to gain some
confidence on the correctness of the properties without paying the
price of formally proving them.  This is the goal of Cryptol's {\tt
  :check}\indCmdCheck command, inspired by Haskell's quick-check
library~\cite{quickcheck}.  Instead of trying to formally prove your
property, {\tt :check} tests it at random values to give you quick
feedback.  This approach is very suitable for rapid development.  By
using automated testing you get frequent and quick updates from
Cryptol regarding the status of your properties, as you work through
your code. If you introduce a bug, it is likely (although not
guaranteed) that the {\tt :check} command will alert you right
away. Once you are satisfied with your code, you can use the {\tt
  :prove} command to conduct the formal proofs, potentially leaving
them running overnight.

The syntax of the {\tt :check} command is precisely the same as the
{\tt :prove} command. By default, it will run your property over 100
randomly generated test cases.

\begin{Exercise}\label{ex:quick:0}
  Use the {\tt :check} command to test the property {\tt
    caesarCorrect} you have defined in
  \exerciseref{ex:cond:2}, for messages of length
  10.  Use the command {\tt :set tests=1000}\indQCCount to change the
  number of test cases to 1,000.  Observe the test coverage statistics
  reported by Cryptol. How is the total number of cases computed?
\end{Exercise}
\begin{Answer}\ansref{ex:quick:0}
  Here is the interaction with Cryptol (when you actually run this,
  you will see the test cases counting up as they are
  performed):\indQCCount
\begin{Verbatim}
  Cryptol> :check (caesarCorrect : ([8], String 10) -> Bit)
  Using random testing.
  passed 100 tests.
  Coverage: 0.00% (100 of 2^^88 values)
  Cryptol> :set tests=1000
  Cryptol> :check (caesarCorrect : ([8], String 10) -> Bit)
  Using random testing.
  passed 1000 tests.
  Coverage: 0.00% (1000 of 2^^88 values)
\end{Verbatim}
In each case, Cryptol tells us that it checked a minuscule portion of
all possible test cases: A good reminder of what {\tt :check} is
really doing.  The number of test cases is: $2^{8+8\times10} =
2^{88}$. We have 8-bits for the {\tt d} value, and $10*8$ bits total
for the 10 characters in {\tt msg}, giving us a total of 88
bits. Since the input is 88 bits wide, we have $2^{88}$ potential test
cases. Note how the number of test cases increase exponentially with
the size of the message.
\end{Answer}

\begin{Exercise}\label{ex:quick:1}
  If the property is \emph{small} in size, {\tt :check} might as well
  prove/disprove it. Try the following commands:
\begin{Verbatim}
  :check True
  :check False
  :check \x -> x==(x:[8])
\end{Verbatim}
\end{Exercise}
\begin{Answer}\ansref{ex:quick:1}
\begin{Verbatim}
  Cryptol> :check True
  Using exhaustive testing.
  passed 1 tests.
  QED
  Cryptol> :check False
  Using exhaustive testing.
  FAILED for the following inputs:
  Cryptol> :check \x -> x == (x:[8])
  Using exhaustive testing.
  passed 256 tests.
  QED
\end{Verbatim}
Note that when Cryptol is able to exhaust all possible inputs, it returns QED, since the property is effectively proven.
\end{Answer}

\begin{Exercise}\label{ex:quick:2}
  Write a bogus property that will be very easy to disprove using {\tt
    :prove}, while {\tt :check} will have a hard time obtaining the
  counterexample.  The moral of this exercise is that you should try
  {\tt :prove} early in your development and not get too comfortable
  with the results of {\tt :check}!\indCmdProve\indCmdCheck
\end{Exercise}
\begin{Answer}\ansref{ex:quick:2}
\begin{code}
  property easyBug x = x != (76123:[64])
\end{code}
The {\tt :prove} command will find the counterexample almost
instantaneously, while {\tt :check} will have a hard time!
\end{Answer}

\paragraph*{Exhaustive testing}
The \texttt{:exhaust} command is a variant of \texttt{:check} that
always uses exhaustive testing: It tests the property on every
possible combination of inputs. The number of tests run by
\texttt{:check} is limited to the value of \texttt{:set tests}, but
\texttt{:exhaust} may run a much larger number. If the argument types
are not sufficiently small, \texttt{:exhaust} may run for longer than
you are willing to wait!

\begin{Exercise}\label{ex:quick:3}
  Try the following command to check a conjecture about arithmetic.
  \begin{Verbatim}
  :check \(x:[16]) -> (x + 6) / 3 == x / 3 + 2
  \end{Verbatim}
  What is the result? Does the response give you much confidence in
  the property? Now try checking the same predicate using
  \texttt{:exhaust}. Does the property hold in general?
\end{Exercise}
\begin{Answer}\ansref{ex:quick:3}
  The predicate passes 100 tests, with a test coverage of 0.15\%.
\begin{Verbatim}
  Cryptol> :exhaust \(x:[16]) -> (x + 6) / 3 == x / 3 + 2
  Using exhaustive testing.
  (\(x : [16]) -> (x + 6) / 3 == x / 3 + 2) 0xfffa = False
\end{Verbatim}
  The property does not hold in general.
\end{Answer}

\paragraph*{Bulk operations} If you use {\tt :check}\indCmdCheck and
{\tt :prove}\indCmdProve commands without any arguments, Cryptol will
check and prove all the properties defined in your program. This is a
simple means of exercising all your properties automatically.

\todo[inline]{Do we want to reintroduce the \texttt{-n} switch, per
  \ticket{276}?}
%% \paragraph*{Automatic invocation} You will notice that Cryptol will
%% automatically {\tt :check} the properties in your file when you
%% load the file from the command line, giving you a quick status
%% report. You can turn this behavior off by passing Cryptol the flag
%% {\tt -n}, like this: {\tt cryptol -n myfile.cry}. You can also
%% interrupt a running {\tt :prove}\indCmdProve or {\tt
%% :check}\indCmdCheck command by pressing {\tt
%% Ctrl-C}.\indCmdAutoCheck

%=====================================================================
% \section{Checking satisfiability}
% \label{sec:sat}
\sectionWithAnswers{Checking satisfiability}{sec:sat}

Closely related to proving properties is the notion of checking
satisfiability. In satisfiability checking,\indCmdSat we would like to find
arguments to a bit-valued function such that it will evaluate to {\tt True},
i.e., it will be satisfied.\indSat

One way to think about satisfiability checking is \emph{intelligently}
searching for a solution. Here is a simple example. Let us assume we
would like to compute the modular square-root of 9 as an 8-bit
value. The obvious solution is {\tt 3}, of course, but we are
wondering if there are other solutions to the equation $x^2 \equiv 9
\imod{2^8}$. To get started, let us first define a function that will
return {\tt True} if its argument is a square-root of 9:
\begin{code}
  isSqrtOf9 : [8] -> Bit
  isSqrtOf9 x = x*x == 9
\end{code}
Any square-root of 9 will make the function {\tt isSqrtOf9} return
{\tt True}, i.e., it will \emph{satisfy} it.  Thus, we can use
Cryptol's satisfiability checker to find those values of {\tt x}
automatically:
\begin{Verbatim}
  Cryptol> :sat isSqrtOf9 
  isSqrtOf9 3 = True
\end{Verbatim}
Not surprisingly, Cryptol told us that 3 is one such value. We can
search for other solutions by explicitly disallowing 3:
\begin{Verbatim}
  Cryptol> :sat \x -> isSqrtOf9 x && ~(elem (x, [3]))
  \x -> isSqrtOf9 x && ~(elem (x, [3])) 131 = True
\end{Verbatim}
Note the use of the \lamex to\indLamExp indicate the new
constraint. (Of course, we could have defined another function {\tt
  isSqrtOf9ButNot3} for the same effect, but the \lamex is really
handy in this case.) We have used the function {\tt elem} you have
defined in \exerciseref{ex:recfun:4:1}\indElem to
express the constraint {\tt x} must not be 3. In response, Cryptol
told us that {\tt 125} is another solution. Indeed $125 * 125 =
9\imod{2^7}$, as you can verify separately.  We can search for more:
\begin{Verbatim}
  Cryptol> :sat \x -> isSqrtOf9 x && ~(elem (x, [3, 125]))
  \x -> isSqrtOf9 x && ~(elem (x, [3, 131])) 253 = True
\end{Verbatim}
Rather than manually adding solutions we have already seen, we can
search for other solutions by asking the satisfiability checker for
more solutions using the {\tt satNum} setting:
\begin{Verbatim}
  Cryptol> :set satNum = 4
  Cryptol> :sat isSqrtOf9
  isSqrtOf9 3 = True
  isSqrtOf9 131 = True
  isSqrtOf9 125 = True
  isSqrtOf9 253 = True
\end{Verbatim}
By default, {\tt satNum} is set to {\tt 1}, so we only see one
solution. When we change it to {\tt 4}, the satisfiability checker
will try to find \emph{up to} 4 solutions. We can also set it to {\tt
  all}, which will try to find as many solutions as possible.
\begin{Verbatim}
  Cryptol> :set satNum = all
  Cryptol> :sat isSqrtOf9
  isSqrtOf9 3 = True
  isSqrtOf9 131 = True
  isSqrtOf9 125 = True
  isSqrtOf9 253 = True
\end{Verbatim}
So, we can rest assured that there are exactly four 8-bit square roots
of 9; namely 3, 131, 125, and 253. (Note that Cryptol can return the
satisfying solutions in any order depending on the backend-solver and
other configurations. What is guaranteed is that you will get
precisely the same set of solutions at the end.)

The whole point of the satisfiability checker is to be able to quickly
search for particular values that are solutions to potentially
complicated bit-valued functions. In this sense, satisfiability
checking can also be considered as an automated way to invert a
certain class of functions, going back from results to arguments. Of
course, this search is not done blindly, but rather using SAT\glosSAT
and SMT\glosSMT solvers to quickly find the satisfying
values. Cryptol's {\tt :sat} command\indCmdSat hides the complexity,
allowing the user to focus on the specification of the problem.

\todo[inline]{Re-add chapter on verification, or two chapters on
  \texttt{:sat} and \texttt{:prove}.} 
% (We will see several larger applications of the satisfiability
% checker in \autoref{chap:usingsat}.)

\begin{Exercise}\label{ex:sat:0}
  Fermat's last theorem states that there are no integer solutions to
  the equation $a^n + b^n = c^n$ when $a, b, c > 0,$ and $n > 2$. We
  cannot code Fermat's theorem in Cryptol since we do not have
  arbitrary integers, but we can code the modular version of it where
  the exponentiation and addition is done modulo a fixed
  bit-size. Write a function {\tt modFermat} with the following
  signature:
\begin{code}
  type Quad a = ([a], [a], [a], [a])
  modFermat : {s} (fin s, s >= 2) => Quad s -> Bit
\end{code}
such that {\tt modFermat (a, b, c, n)} will return {\tt True} if the
modular version of Fermat's equation is satisfied by the values of
{\tt a}, {\tt b}, {\tt c}, and {\tt n}. Can you explain why you need
the constraints {\tt fin s} and {\tt s >= 2}?
\end{Exercise}
\begin{Answer}\ansref{ex:sat:0}
\begin{code}
  modFermat (a, b, c, n) = (a > 0) && (b > 0) && (c > 0) && (n > 2)
                         && (a^^n + b^^n == c^^n)
\end{code}
The {\tt fin s} predicate comes from the fact that we are doing
arithmetic and comparisons. The predicate {\tt s >= 2} comes from the
fact that we are comparing {\tt n} to {\tt 2}, which needs at least
$2$ bits to represent.
\end{Answer}

\begin{Exercise}\label{ex:sat:1}
  Use the {\tt :sat} command to see if there are any satisfying values
  for the modular version of Fermat's last theorem for various bit
  sizes.  Surprised? What can you conclude from your observations?
\end{Exercise}
\begin{Answer}\ansref{ex:sat:1}
We can try different instantiations as follows:
\begin{Verbatim}
  Cryptol> :sat modFermat : Quad(2) -> Bit
  modFermat : Quad(2) -> Bit (1, 2, 1, 3) = True
  Cryptol> :sat modFermat : Quad(3) -> Bit
  modFermat : Quad(3) -> Bit (4, 4, 4, 4) = True
  Cryptol> :sat modFermat : Quad(4) -> Bit
  modFermat : Quad(4) -> Bit (4, 4, 4, 8) = True
\end{Verbatim}
The modular form of Fermat's last theorem does not hold for any of the
instances up to and including 12-bits wide, when I stopped
experimenting myself. It is unlikely that it will hold for any
particular bit-size, although the above demonstration is not a proof.
(We would need to consult a mathematician for the general result!)
Also note that Cryptol takes longer and longer to find a satisfying
instance as you increase the bit-size.
\end{Answer}

%=====================================================================
%% \section{Safety checking}
%% \label{sec:safetychecking}

%% Safety checking is all about run-time exceptions: Division-by-zero
%% or index-out-of-bounds being the most infamous ones. Unlike many
%% other languages, Cryptol does not suffer from a certain class of
%% safety violations, such as trying to put in the number 1000 in a
%% buffer that is only 8-bits wide, or trying to multiply a string
%% with an integer; but certain run-time exceptions are still
%% possible. The goal of \emph{safety checking} is to ensure that these
%% exceptions cannot happen at run-time.\indSafe\indCmdSafe

%% Here is a simple example to demonstrate:
%% \begin{Verbatim}
%% Cryptol> :safe \(x, y) -> x/y:[8]
%% *** 1 safety condition to be checked.
%% *** Checking for violations of: ``"command line", line 1, col 13: divide by zero''
%% *** Violation detected:
%% (\(x, y) -> x/y:[8]) (0, 0)
%%          = "command line", line 1, col 14: divide by zero
%% *** 1 problem found.
%%\end{Verbatim}
%%Cryptol is telling us that the function {\tt $\backslash$(x, y) ->
%% x/y} is \emph{not} safe. When given the argument {\tt (0, 0)} it
%% will cause a division by zero exception. If we properly guard for
%% the condition, then the safety check will indeed pass:
%%\begin{Verbatim}
%% Cryptol> :safe \(x, y) -> if y == 0 then 0 else x/y:[8]
%% *** 1 safety condition to be checked.
%% *** Checking for violations of: ``"command line", line 1, col 35: divide by zero''
%% *** Verified safe.
%% *** All safety checks pass, safe to execute
%%\end{Verbatim}
%%By using {\tt :safe}\indCmdSafe on our top-level functions we can
%% ensure that our program is guaranteed not to have any run-time
%% exceptions.  Cryptol's safety-checker will detect the following
%% run-time exceptions:
%%\begin{itemize}
%% \item Division ({\tt /}) and modulus ({\tt \%}) by 0, both modular
%%   and polynomial versions (See
%%   \autoref{sec:polynomials})\indDiv\indMod\indDivPoly\indModPoly
%% \item Index-out-of-bounds errors for {\tt @}, {\tt @@}, {\tt !} and
%%   {\tt !!},\indIndex\indIndexs\indRIndex\indRIndexs
%%  \item Calling {\tt lg2} on {\tt 0},\indLg
%%  \item Data dependence on {\tt error} and the {\tt undefined}
%%    values,\indError\indUndefined
%%  \item Failure cases for Cryptol's {\tt ASSERT}
%%    expression.\indAssert
%%\end{itemize}

%%\begin{Exercise}\label{ex:safe:0}
%%Consider the function:
%%\begin{code}
%%  lkup : ([10][8], [4]) -> [8]
%%  lkup (xs, i) = xs @ i
%%\end{code}
%%Is it safe? For what values of {\tt i} does it throw an exception?
%%\end{Exercise}
%%\begin{Answer}\ansref{ex:safe:0}
%%  {\tt lkup} will throw an exception whenever {\tt i} is at least
%%  {\tt 10}. We can use Cryptol to detect this safety violation:
%%\begin{Verbatim}
%%  Cryptol> :safe lkup
%%  *** 1 safety condition to be checked.
%%  *** Checking for violations of: ``..: index of symbolic-value is out of bounds
%%  (valid range is 0 thru 9).''
%%  *** Violation detected:
%%  lkup ([255 255 255 255 255 255 255 255 255 255], 13)
%%           = .. index of 13 is out of bounds
%%  (valid range is 0 thru 9).
%%  *** 1 problem found.
%%\end{Verbatim}
%%In this case Cryptol identified that there will be an exception when
%% {\tt i} is 13.
%%\end{Answer}

%%\begin{Exercise}\label{ex:safe:1}
%%Here is an attempt to make {\tt lkup} safe:
%%\begin{code}
%%  lkup1 : ([10][8], [4]) -> [8]
%%  lkup1 (xs, i) = if i > 10 then 0 else xs @ i
%%\end{code}
%%Is the fix successful? Use the {\tt :safe} command to check your
%% answer.
%%\end{Exercise}
%%\begin{Answer}\ansref{ex:safe:1}
%%No. {\tt lkup1} suffers from the infamous off-by-1 error!
%%\begin{Verbatim}
%%  Cryptol> :safe lkup1
%%  *** 1 safety condition to be checked.
%%  *** Checking for violations of: ``..: index of symbolic-value is out of bounds
%%  (valid range is 0 thru 9).''
%%  *** Violation detected:
%%  lkup1 ([255 255 255 255 255 255 255 255 255 255], 10)
%%           = ..: index of 10 is out of bounds
%%  (valid range is 0 thru 9).
%%  *** 1 problem found.
%%\end{Verbatim}
%%This time Cryptol will pin-point to the only potential error value,
%% where {\tt i} is 10.
%%\end{Answer}
%%
%%\begin{Exercise}\label{ex:safe:2}
%%  Write a version of {\tt lkup} that fixes the safety issue. Use the
%%  {\tt :safe} command to verify that your fix is good.
%%\end{Exercise}
%%\begin{Answer}\ansref{ex:safe:2}
%%\begin{code}
%%  lkup2 : ([10][8], [4]) -> [8]
%%  lkup2 (xs, i) = if i > 9 then 0 else xs @ i
%%\end{code}
%%We have:
%%\begin{Verbatim}
%%  Cryptol> :safe lkup2
%%  *** 1 safety condition to be checked.
%%  *** Checking for violations of: ``..: index of symbolic-value is out of bounds
%%  (valid range is 0 thru 9).''
%%  *** Verified safe.
%%  *** All safety checks pass, safe to execute.
%%\end{Verbatim}
%%\end{Answer}
%%
%%\paragraph*{Over-conservative safety checking}
%% Cryptol's safety checker is very conservative: If it says that a
%% function is \emph{safe to execute}, you are guaranteed that the
%% execution can not raise any run-time exception. However, the
%% converse is not always true: In certain cases, Cryptol can signal a
%% potential safety violation which will actually not cause an error
%% when \emph{run in the normal interpreter}.
%%
%% Here is an example to illustrate the idea. (The example is somewhat
%% contrived, but similar patterns of coding does arise in Cryptol
%% programming.) We will use Cryptol's {\tt undefined}
%% value\indUndefined, which would throw an exception if executed:
%%\begin{Verbatim}
%%  Cryptol> (1:[8]) + undefined 
%%  Entered Cryptol 'undefined' value
%%\end{Verbatim}
%%
%%Cryptol's {\tt undefined}\indUndefined is useful when representing
%% values that should \emph{not} be needed during execution, as
%% illustrated below:
%%\begin{code}
%%  choose : [8] -> [8]
%%  choose c = ([undefined] # [1 ..]) @ index
%%    where index = if c == 0 then 1 else c
%%\end{code}
%%Notice that {\tt undefined} will not be referenced: The {\tt index}
%% can never be {\tt 0}. (Why?) If we ask {\tt :safe}, though, it will
%% tell us the following:
%%\begin{Verbatim}
%%  Cryptol> :safe choose
%%  *** 1 safety condition to be checked.
%%  *** Checking for violations of: ``Entered Cryptol 'undefined' value''
%%  *** Violation detected:
%%  choose 0
%%           = 1
%%  *** ALERT: This safety issue is only relevant for certain target platforms
%%  ***        and hence does not cause an exception in the above run.
%%  ***        C and hardware translations might be subject to stricter semantics.
%%  *** 1 problem found.
%%\end{Verbatim}
%%The {\tt :safe} command\indCmdSafe told us that there is a safety
%% violation when {\tt c} is {\tt 0}, then it also showed us that if
%% we run {\tt choose} with {\tt 0} we get the value {\tt 1}, not an
%% exception. At first, this sounds quite counterintuitive. However,
%% the {\tt ALERT} given by Cryptol sheds some light into the
%% potential problem.  When Cryptol is used in the normal interpreter
%% mode, this program will indeed not produce any exceptions. However,
%% Cryptol programs can also be compiled to other targets, such as C
%% or Java, or other hardware platforms. What Cryptol is telling us is
%% that these translations are \emph{not} guaranteed to be exception
%% free, and hence it is worth making sure the generated code will
%% behave properly.
%%
%% While it is important to understand what Cryptol is telling us, the
%% user can safely ignore these alerts most of the time, especially
%% when one is focusing only at the Cryptol level. (We will revisit
%% this topic in \autoref{sec:pitfall:thmexceptions}.)
%%
%%\begin{Exercise}\label{ex:safe:3}
%%  Rewrite {\tt choose} so that {\tt :safe} will not issue any
%%  alerts.
%%\end{Exercise}
%%\begin{Answer}\ansref{ex:safe:3}
%%  One way would be to replace {\tt undefined}\indUndefined with {\tt
%%  zero}.\indZero Since the result will never be needed, we can
%%  safely replace it with {\tt 0} without changing the semantics.
%%\end{Answer}

%%% Local Variables: 
%%% mode: latex
%%% TeX-master: "../main/Cryptol"
%%% End: 
