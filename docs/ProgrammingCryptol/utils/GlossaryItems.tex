%%% NB. If you put a citation here, make sure it appears elsewhere in
%%% the document too, otherwise bibtex won't be able to find it!
\newcommand{\glosAES}{\glossary{name=AES, description={The Advanced
Encryption Standard~\cite{aes}}}\xspace}

\newcommand{\glosFibonacci}{\glossary{name=Fibonacci numbers,
description={The sequence $0, 1, 1, 2, 3, 5, \ldots$ After the
elements $0$ and $1$, each consecutive element is the sum of the two
previous numbers~\cite{wiki:fibonacci}}}\xspace}

\newcommand{\glosPlaintext}{\glossary{name=Plaintext, description={A
``readable'' message that we would like to encrypt, the message in the
clear}}\xspace}

\newcommand{\glosCiphertext}{\glossary{name=Ciphertext,
description={The result of encrypting a plaintext message,
``unreadable'' unless the key is known}}\xspace}

\newcommand{\glosCipherkey}{\glossary{name=Cipherkey, description={The
key used in a particular encryption/decryption task}}\xspace}

\newcommand{\glosSAT}{\glossary{name=SAT Solver, description={An
automated tool for solving boolean satisfiability problems. Cryptol
uses SAT/SMT solvers in order to provide its high-assurance
capabilities}}\xspace}

\newcommand{\glosSMT}{\glossary{name=SMT Solver,
description={Satisfiability Modulo Theories: An automated tool for
establishing satisfiability problems with respect to certain
theories. One of the theories of interest to Cryptol is that of
bit-vectors, as it provides a natural medium for translating Cryptol's
bit-precise theorems}}\xspace}

\newcommand{\glosNIST}{\glossary{name=NIST, description={National
Institute of Standards and Technology. The institution in charge of
standardizing cryptoalgorithms (amongst many other things) in
USA.}}\xspace}

\newcommand{\glosGF}{\glossary{name=GF, description={Galois Field. The
notation GF($p^n$) stands for the finite field with $p^n$
elements. For instance, the AES algorithm uses GF($2^8$)
internally~\cite{wiki:galoisfield}}}\xspace}

%%% Local Variables: 
%%% mode: latex
%%% TeX-master: "../main/Cryptol"
%%% End: 
